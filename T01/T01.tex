\documentclass{article}
\usepackage{graphicx}
\usepackage{amsmath}
\usepackage{amssymb}
\usepackage{amsfonts}

\title{MEMAD-T01}
\author{ALEJANDRO ZARATE MACIAS}
\date{25 de Agosto 2025}

\begin{document}

\maketitle

% ========================================
% INTRODUCCIÓN
% ========================================
\section*{Introducción}

Para esta tarea se busca la resolución de problemas relacionados con álgebra vectorial, cálculo multivariable y métodos de optimización.
La idea es utilizar lo aprendido en los videos proporcionados como material de estudio, además de otras

% ========================================
% SECCIÓN 1
% ========================================
\section{Problema 1}

\subsection{Enunciado}
Encontrar los ángulos (ambos) entre los vectores $\vec{p} = (-1, 3, 1)$ y $\vec{q} = (-2, -3, -7)$.

\subsection{Metodología}
Para la resolución de este problema, se debe utilizar la fórmula

\begin{align}
\cos(\theta) = \frac{<\vec{p},\vec{q}>}{||\vec{p}|| ||\vec{q}||}
\end{align}

donde $<\vec{p},\vec{q}>$ es el producto punto de los vectores y $||\vec{p}||$ y $||\vec{q}||$ son las normas de los vectores.
Para esto, se deben seguir los siguientes pasos:
\begin{enumerate}
    \item[-]  Calcular el producto punto $<\vec{p},\vec{q}>$.
    \item[-] Calcular las normas $||\vec{p}||$ y $||\vec{q}||$.
    \item[-] Sustituir los valores en la fórmula.
    \item[-] Despejar $\theta$.
    \item[-] Calcular los ángulos.
\end{enumerate}

\subsection{Resultados}
\setcounter{equation}{0}

Siguiendo los pasos mencionados, en la metodología, tenemos que el producto punto es:
\begin{align}
<\vec{p},\vec{q}> &= (-1)(-2) + (3)(-3) + (1)(-7) \\
&= 2 - 9 - 7 \\
&= -14
\end{align}

Ahora calculamos las normas.
Para $\vec{p}$:
\begin{align}
||\vec{p}|| &= \sqrt{(-1)^2 + 3^2 + 1^2} \\
&= \sqrt{1 + 9 + 1} \\
&= \sqrt{11}
\end{align}

Y para $\vec{q}$:
\begin{align}
||\vec{q}|| &= \sqrt{(-2)^2 + (-3)^2 + (-7)^2} \\
&= \sqrt{4 + 9 + 49} \\
&= \sqrt{62}
\end{align}

Sustituyendo los valores en la fórmula, tenemos:
\begin{align}
\cos(\theta) &= \frac{-14}{(\sqrt{11}) (\sqrt{62})} \\
&= \frac{-14}{\sqrt{682}} 
\end{align}

Despejando $\theta$, tenemos:
\begin{align}
\theta = \arccos(\frac{-14}{\sqrt{682}}) 
\end{align}

Por lo tanto:
\begin{align}
\theta_1 &= \arccos(-0.536) \\
&\approx 122.41^\circ
\end{align}

Como resultado dio mayor a $90\circ$, este es el ángulo obtuso entre los vectores.

El ángulo agudo correspondiente es:
\begin{align}
\theta_2 &= 180^\circ - 122.41^\circ \\
&= 57.59^\circ
\end{align}

\subsection{Discusión}

\subsection{Conclusión}

% ========================================
% SECCIÓN 2
% ========================================
\section{Problema 2}

\subsection{Enunciado}
Encontrar dos vectores no paralelos que sean ortogonales a $\vec{p} = (1, 1, 1)$.

\subsection{Metodología}

Para encontrar vectores ortogonales a $\vec{p} = (1, 1, 1)$, debemos encontrar vectores $\vec{v}$ tales que el producto punto $\vec{p} \cdot \vec{v} = 0$.

Si $\vec{v} = (x, y, z)$, entonces:
\begin{align}
\vec{p} \cdot \vec{v} = (1, 1, 1) \cdot (x, y, z) = x + y + z = 0
\end{align}

Para verificar que los vectores no sean paralelos, utilizaremos el concepto de proporcionalidad: dos vectores $\vec{v_1} = (a_1, b_1, c_1)$ y $\vec{v_2} = (a_2, b_2, c_2)$ son paralelos si y solo si las razones entre sus componentes correspondientes son iguales, es decir:
$$\frac{a_1}{a_2} = \frac{b_1}{b_2} = \frac{c_1}{c_2}$$

Entonces, para encontrar dos vectores no paralelos que sean ortogonales a $\vec{p}$, seguiremos estos pasos:
\begin{enumerate}
    \item[-] Elegir vectores que cumplan la condición $x + y + z = 0$.
    \item[-] Verificar que ambos sean ortogonales a $\vec{p}$.
    \item[-] Verificar que no sean paralelos usando el concepto de proporcionalidad.
\end{enumerate}

\subsection{Resultados}
\setcounter{equation}{0}

Vector 1: $\vec{v_1} = (5,-1,-4)$ \\
Verificación de ortogonalidad: \\
\begin{align}
\vec{p} \cdot \vec{v_1} &= (1, 1, 1) \cdot (5, -1, -4) \\
&= 1(5) + 1(-1) + 1(-4) \\
&= 5 - 1 - 4 \\
&= 0
\end{align}

Vector 2: $\vec{v_2} = (1, 2, -3)$ \\
Verificación de ortogonalidad: \\
\begin{align}
\vec{p} \cdot \vec{v_2} &= (1, 1, 1) \cdot (1, 2, -3) \\
&= 1(1) + 1(2) + 1(-3) \\
&= 1 + 2 - 3 \\
&= 0
\end{align}

Ahora debemos asegurarnos de que los vectores no sean paralelos, para esto usamos proporcionalidad:
\begin{align}
\frac{5}{1} &= 5 \\
\frac{-1}{2} &= -0.5 \\
\frac{-4}{-3} &\approx 1.33
\end{align}

Como $5 \neq -0.5 \neq 1.33$, los vectores no son paralelos.

\subsection{Discusión}

\subsection{Conclusión}

% ========================================
% SECCIÓN 3
% ========================================
\section{Problema 3}

\subsection{Enunciado}
Sea $n$ un número natural y $A$ como sigue:
$$A = \begin{vmatrix}
n & n+1 & n+2 \\
n+3 & n+4 & n+5 \\
n+6 & n+7 & n+8
\end{vmatrix}$$

Mostrar que $\det(A)$ permanece constante con respecto a $n$.

\subsection{Metodología}

Para demostrar que $\det(A)$ permanece constante con respecto a $n$, calcularemos el determinante para un valor específico de $n$ y luego analizaremos la estructura de la matriz para mostrar dependencia lineal entre las filas.

\subsection{Resultados}
\setcounter{equation}{0}

Calculemos el determinante para $n = 1$:
$$A = \begin{vmatrix}
1 & 2 & 3 \\
4 & 5 & 6 \\
7 & 8 & 9
\end{vmatrix}$$

Expandiendo por la primera fila:
\begin{align}
\det(A) &= 1 \begin{vmatrix} 5 & 6 \\ 8 & 9 \end{vmatrix} - 2 \begin{vmatrix} 4 & 6 \\ 7 & 9 \end{vmatrix} + 3 \begin{vmatrix} 4 & 5 \\ 7 & 8 \end{vmatrix} \\
&= 1 (5 \times 9 - 6 \times 8) - 2 (4 \times 9 - 6 \times 7) + 3 (4 \times 8 - 5 \times 7) \\
&= 1 (45 - 48) - 2 (36 - 42) + 3 (32 - 35) \\
&= 1 (-3) - 2 (-6) + 3 (-3) \\
&= -3 + 12 - 9 \\
&= 0
\end{align}

El hecho de que $\det(A) = 0$ nos indica una posible dependencia lineal entre las filas.

Analicemos la estructura de la matriz. Observemos que cada fila es una progresión aritmética con diferencia común 1, y entre filas consecutivas hay una diferencia común de 3 en cada posición:

Para la matriz general:
$$\begin{pmatrix}
n & n+1 & n+2 \\
n+3 & n+4 & n+5 \\
n+6 & n+7 & n+8
\end{pmatrix}$$

Notemos que la tercera fila menos la segunda fila es igual a la segunda fila menos la primera fila:
\begin{align}
\text{Fila 3} - \text{Fila 2} &= (3, 3, 3) \\
\text{Fila 2} - \text{Fila 1} &= (3, 3, 3)
\end{align}

Por lo tanto: $\text{Fila 1} - 2 \times \text{Fila 2} + \text{Fila 3} = (0, 0, 0)$

Esta relación lineal entre las filas demuestra que son linealmente dependientes, lo que implica que $\det(A) = 0$ para cualquier valor de $n$.

\subsection{Discusión}

\subsection{Conclusión}

% ========================================
% SECCIÓN 4
% ========================================
\section{Problema 4}

\subsection{Enunciado}
Resolver el siguiente sistema de ecuaciones lineales usando determinantes:
\begin{align}
x + y + z &= 1 \\
2y - 2z - w &= -7 \\
x + y - z &= -3 \\
x + z + w &= 2
\end{align}

\subsection{Metodología}

Para resolver el sistema de ecuaciones lineales usando determinantes, utilizaremos el método de Cramer. El sistema se puede escribir en forma matricial como $A\mathbf{x} = \mathbf{b}$, donde:

\begin{align*}
    A = \begin{pmatrix}
    1 & 1 & 1 & 0 \\
    0 & 2 & -2 & -1 \\
    1 & 1 & -1 & 0 \\
    1 & 0 & 1 & 1
    \end{pmatrix}, \quad \mathbf{x} = \begin{pmatrix} x \\ y \\ z \\ w \end{pmatrix}, \quad \mathbf{b} = \begin{pmatrix} 1 \\ -7 \\ -3 \\ 2 \end{pmatrix}
\end{align*}

El método de Cramer establece que:

\begin{align*}
    x = \frac{\det(A_x)}{\det(A)}, \quad y = \frac{\det(A_y)}{\det(A)}, \quad z = \frac{\det(A_z)}{\det(A)}, \quad w = \frac{\det(A_w)}{\det(A)}    
\end{align*}

donde $A_x$, $A_y$, $A_z$, $A_w$ son las matrices obtenidas al reemplazar la columna correspondiente de $A$ por el vector $\mathbf{b}$.

Los pasos a seguir son:
\begin{enumerate}
    \item[-] Calcular $\det(A)$.
    \item[-] Calcular $\det(A_x)$, $\det(A_y)$, $\det(A_z)$, $\det(A_w)$.
    \item[-] Aplicar el método de Cramer para obtener cada variable.
\end{enumerate}

\subsection{Resultados}
\setcounter{equation}{0}

Primero calculamos $\det(A)$ expandiendo por la primera fila:

\begin{align}
    det(A) = \begin{vmatrix}
    1 & 1 & 1 & 0 \\
    0 & 2 & -2 & -1 \\
    1 & 1 & -1 & 0 \\
    1 & 0 & 1 & 1
    \end{vmatrix}
\end{align}

Quedando de la siguiente manera:
\begin{align}
    det(A) = 1 \cdot \begin{vmatrix}
    2 & -2 & -1 \\
    1 & -1 & 0 \\
    0 & 1 & 1
    \end{vmatrix} - 1 \cdot \begin{vmatrix}
    0 & -2 & -1 \\
    1 & -1 & 0 \\
    1 & 1 & 1
    \end{vmatrix} + 1 \cdot \begin{vmatrix}
    0 & 2 & -1 \\
    1 & 1 & 0 \\
    1 & 0 & 1
    \end{vmatrix}
\end{align}

En este caso, el cuarto elemento de la primera fila fue omitido, ya que al ser $0$, su multiplicación dará $0$. Además, podemos notar que los demás elementos de la primera fila son $1$, por lo que podemos simplificar la expresión a: \\

\begin{align}
    det(A) = \begin{vmatrix}
    2 & -2 & -1 \\
    1 & -1 & 0 \\
    0 & 1 & 1
    \end{vmatrix} - \begin{vmatrix}
    0 & -2 & -1 \\
    1 & -1 & 0 \\
    1 & 1 & 1
    \end{vmatrix} + \begin{vmatrix}
    0 & 2 & -1 \\
    1 & 1 & 0 \\
    1 & 0 & 1
    \end{vmatrix}
\end{align}

Para calcular los 3 determinantes de las matrices resultantes, expandiremos nuevamente por la primer fila de cada matriz.

Primer determinante:
\begin{align}
    \begin{vmatrix}
    2 & -2 & -1 \\
    1 & -1 & 0 \\
    0 & 1 & 1
    \end{vmatrix} &= 2 \begin{vmatrix} -1 & 0 \\ 1 & 1 \end{vmatrix} - (-2) \begin{vmatrix} 1 & 0 \\ 0 & 1 \end{vmatrix} + (-1) \begin{vmatrix} 1 & -1 \\ 0 & 1 \end{vmatrix} \\
    &= 2(-1) + 2(1) - 1(1) \\
    &= -2 + 2 - 1 \\
    &= -1
\end{align}

Segundo determinante:
\begin{align}
    \begin{vmatrix}
    0 & -2 & -1 \\
    1 & -1 & 0 \\
    1 & 1 & 1
    \end{vmatrix} &= 0 - (-2) \begin{vmatrix} 1 & 0 \\ 1 & 1 \end{vmatrix} + (-1) \begin{vmatrix} 1 & -1 \\ 1 & 1 \end{vmatrix} \\
    &= 0 + 2(1) - 1(2) \\
    &= 2 - 2 \\
    &= 0
\end{align}

Tercer determinante:
\begin{align}
    \begin{vmatrix}
    0 & 2 & -1 \\
    1 & 1 & 0 \\
    1 & 0 & 1
    \end{vmatrix} &= 0 - 2 \begin{vmatrix} 1 & 0 \\ 1 & 1 \end{vmatrix} + (-1) \begin{vmatrix} 1 & 1 \\ 1 & 0 \end{vmatrix} \\
    &= 0 - 2(1) - 1(-1) \\
    &= -2 + 1 \\
    &= -1
\end{align}

Por lo tanto:
\begin{align}
    \det(A) &= (-1) - (0) + (-1) \\
    &= -2
\end{align}

Ahora que ya conocemos el determinante de la matriz $A$, podemos calcular los determinantes de las matrices $A_x$, $A_y$, $A_z$ y $A_w$.

Calculamos $\det(A_x)$:

\begin{align}
    A_x = \begin{pmatrix}
    1 & 1 & 1 & 0 \\
    -7 & 2 & -2 & -1 \\
    -3 & 1 & -1 & 0 \\
    2 & 0 & 1 & 1
    \end{pmatrix}
\end{align}

Expandiendo por la primera fila:
\begin{align}
    \det(A_x) &= 1 \cdot \begin{vmatrix} 2 & -2 & -1 \\ 1 & -1 & 0 \\ 0 & 1 & 1 \end{vmatrix} - 1 \cdot \begin{vmatrix} -7 & -2 & -1 \\ -3 & -1 & 0 \\ 2 & 1 & 1 \end{vmatrix} + 1 \cdot \begin{vmatrix} -7 & 2 & -1 \\ -3 & 1 & 0 \\ 2 & 0 & 1 \end{vmatrix}
\end{align}

Calculamos cada determinante 3×3:

Primer determinante:
\begin{align}
    \begin{vmatrix} 2 & -2 & -1 \\ 1 & -1 & 0 \\ 0 & 1 & 1 \end{vmatrix} &= 2 \begin{vmatrix} -1 & 0 \\ 1 & 1 \end{vmatrix} - (-2) \begin{vmatrix} 1 & 0 \\ 0 & 1 \end{vmatrix} + (-1) \begin{vmatrix} 1 & -1 \\ 0 & 1 \end{vmatrix} \\
    &= 2(-1) + 2(1) - 1(1) \\
    &= -2 + 2 - 1 \\
    &= -1
\end{align}

Segundo determinante:
\begin{align}
    \begin{vmatrix} -7 & -2 & -1 \\ -3 & -1 & 0 \\ 2 & 1 & 1 \end{vmatrix} &= -7 \begin{vmatrix} -1 & 0 \\ 1 & 1 \end{vmatrix} - (-2) \begin{vmatrix} -3 & 0 \\ 2 & 1 \end{vmatrix} + (-1) \begin{vmatrix} -3 & -1 \\ 2 & 1 \end{vmatrix} \\
    &= -7(-1) + 2(-3) - 1(-3 + 2) \\
    &= 7 - 6 - 1(-1) \\
    &= 7 - 6 + 1 \\
    &= 2
\end{align}

Tercer determinante:
\begin{align}
    \begin{vmatrix} -7 & 2 & -1 \\ -3 & 1 & 0 \\ 2 & 0 & 1 \end{vmatrix} &= -7 \begin{vmatrix} 1 & 0 \\ 0 & 1 \end{vmatrix} - 2 \begin{vmatrix} -3 & 0 \\ 2 & 1 \end{vmatrix} + (-1) \begin{vmatrix} -3 & 1 \\ 2 & 0 \end{vmatrix} \\
    &= -7(1) - 2(-3) - 1(0 - 2) \\
    &= -7 + 6 - 1(-2) \\
    &= -7 + 6 + 2 \\
    &= 1
\end{align}

Por lo tanto:
\begin{align}
    \det(A_x) &= 1(-1) - 1(2) + 1(1) - 0 \\
    &= -1 - 2 + 1 \\
    &= -2
\end{align}

Calculamos $\det(A_y)$:

\begin{align}
    A_y = \begin{pmatrix}
    1 & 1 & 1 & 0 \\
    0 & -7 & -2 & -1 \\
    1 & -3 & -1 & 0 \\
    1 & 2 & 1 & 1
    \end{pmatrix}
\end{align}

Expandiendo por la primera fila:
\begin{align}
    \det(A_y) &= 1 \cdot \begin{vmatrix} -7 & -2 & -1 \\ -3 & -1 & 0 \\ 2 & 1 & 1 \end{vmatrix} - 1 \cdot \begin{vmatrix} 0 & -2 & -1 \\ 1 & -1 & 0 \\ 1 & 1 & 1 \end{vmatrix} + 1 \cdot \begin{vmatrix} 0 & -7 & -1 \\ 1 & -3 & 0 \\ 1 & 2 & 1 \end{vmatrix}
\end{align}

Calculamos cada determinante 3×3:

Primer determinante:
\begin{align}
    \begin{vmatrix} -7 & -2 & -1 \\ -3 & -1 & 0 \\ 2 & 1 & 1 \end{vmatrix} &= -7 \begin{vmatrix} -1 & 0 \\ 1 & 1 \end{vmatrix} - (-2) \begin{vmatrix} -3 & 0 \\ 2 & 1 \end{vmatrix} + (-1) \begin{vmatrix} -3 & -1 \\ 2 & 1 \end{vmatrix} \\
    &= -7(-1) + 2(-3) - 1(-3 + 2) \\
    &= 7 - 6 + 1 \\
    &= 2
\end{align}

Segundo determinante:
\begin{align}
    \begin{vmatrix} 0 & -2 & -1 \\ 1 & -1 & 0 \\ 1 & 1 & 1 \end{vmatrix} &= 0 - (-2) \begin{vmatrix} 1 & 0 \\ 1 & 1 \end{vmatrix} + (-1) \begin{vmatrix} 1 & -1 \\ 1 & 1 \end{vmatrix} \\
    &= 0 + 2(1) - 1(1 + 1) \\
    &= 2 - 2 \\
    &= 0
\end{align}

Tercer determinante:
\begin{align}
    \begin{vmatrix} 0 & -7 & -1 \\ 1 & -3 & 0 \\ 1 & 2 & 1 \end{vmatrix} &= 0 - (-7) \begin{vmatrix} 1 & 0 \\ 1 & 1 \end{vmatrix} + (-1) \begin{vmatrix} 1 & -3 \\ 1 & 2 \end{vmatrix} \\
    &= 0 + 7(1) - 1(2 + 3) \\
    &= 7 - 5 \\
    &= 2
\end{align}

Por lo tanto:
\begin{align}
    \det(A_y) &= 1(2) - 1(0) + 1(2) \\
    &= 2 + 0 + 2 \\
    &= 4
\end{align}

Calculamos $\det(A_z)$:

\begin{align}
    A_z = \begin{pmatrix}
    1 & 1 & 1 & 0 \\
    0 & 2 & -7 & -1 \\
    1 & 1 & -3 & 0 \\
    1 & 0 & 2 & 1
    \end{pmatrix}
\end{align}

Expandiendo por la primera fila:
\begin{align}
    \det(A_z) &= 1 \cdot \begin{vmatrix} 2 & -7 & -1 \\ 1 & -3 & 0 \\ 0 & 2 & 1 \end{vmatrix} - 1 \cdot \begin{vmatrix} 0 & -7 & -1 \\ 1 & -3 & 0 \\ 1 & 2 & 1 \end{vmatrix} + 1 \cdot \begin{vmatrix} 0 & 2 & -1 \\ 1 & 1 & 0 \\ 1 & 0 & 1 \end{vmatrix}
\end{align}

Calculamos cada determinante 3×3:

Primer determinante:
\begin{align}
    \begin{vmatrix} 2 & -7 & -1 \\ 1 & -3 & 0 \\ 0 & 2 & 1 \end{vmatrix} &= 2 \begin{vmatrix} -3 & 0 \\ 2 & 1 \end{vmatrix} - (-7) \begin{vmatrix} 1 & 0 \\ 0 & 1 \end{vmatrix} + (-1) \begin{vmatrix} 1 & -3 \\ 0 & 2 \end{vmatrix} \\
    &= 2(-3) + 7(1) - 1(2) \\
    &= -6 + 7 - 2 \\
    &= -1
\end{align}

Segundo determinante:
\begin{align}
    \begin{vmatrix} 0 & -7 & -1 \\ 1 & -3 & 0 \\ 1 & 2 & 1 \end{vmatrix} &= 0 - (-7) \begin{vmatrix} 1 & 0 \\ 1 & 1 \end{vmatrix} + (-1) \begin{vmatrix} 1 & -3 \\ 1 & 2 \end{vmatrix} \\
    &= 0 + 7(1) - 1(2 + 3) \\
    &= 7 - 5 \\
    &= 2
\end{align}

Tercer determinante:
\begin{align}
    \begin{vmatrix} 0 & 2 & -1 \\ 1 & 1 & 0 \\ 1 & 0 & 1 \end{vmatrix} &= 0 - 2 \begin{vmatrix} 1 & 0 \\ 1 & 1 \end{vmatrix} + (-1) \begin{vmatrix} 1 & 1 \\ 1 & 0 \end{vmatrix} \\
    &= 0 - 2(1) - 1(0 - 1) \\
    &= -2 + 1 \\
    &= -1
\end{align}

Por lo tanto:
\begin{align}
    \det(A_z) &= 1(-1) - 1(2) + 1(-1) \\
    &= -1 - 2 - 1 \\
    &= -4
\end{align}

Calculamos $\det(A_w)$:

\begin{align}
    A_w = \begin{pmatrix}
    1 & 1 & 1 & 1 \\
    0 & 2 & -2 & -7 \\
    1 & 1 & -1 & -3 \\
    1 & 0 & 1 & 2
    \end{pmatrix}
\end{align}

Expandiendo por la primera fila:
\begin{align}
    \det(A_w) &= 1 \cdot \begin{vmatrix} 2 & -2 & -7 \\ 1 & -1 & -3 \\ 0 & 1 & 2 \end{vmatrix} - 1 \cdot \begin{vmatrix} 0 & -2 & -7 \\ 1 & -1 & -3 \\ 1 & 1 & 2 \end{vmatrix} + 1 \cdot \begin{vmatrix} 0 & 2 & -7 \\ 1 & 1 & -3 \\ 1 & 0 & 2 \end{vmatrix} - 1 \cdot \begin{vmatrix} 0 & 2 & -2 \\ 1 & 1 & -1 \\ 1 & 0 & 1 \end{vmatrix}
\end{align}

Calculamos cada determinante 3×3:

Primer determinante:
\begin{align}
    \begin{vmatrix} 2 & -2 & -7 \\ 1 & -1 & -3 \\ 0 & 1 & 2 \end{vmatrix} &= 2 \begin{vmatrix} -1 & -3 \\ 1 & 2 \end{vmatrix} - (-2) \begin{vmatrix} 1 & -3 \\ 0 & 2 \end{vmatrix} + (-7) \begin{vmatrix} 1 & -1 \\ 0 & 1 \end{vmatrix} \\
    &= 2(-2 + 3) + 2(2) - 7(1) \\
    &= 2 + 4 - 7 \\
    &= -1
\end{align}

Segundo determinante:
\begin{align}
    \begin{vmatrix} 0 & -2 & -7 \\ 1 & -1 & -3 \\ 1 & 1 & 2 \end{vmatrix} &= 0 - (-2) \begin{vmatrix} 1 & -3 \\ 1 & 2 \end{vmatrix} + (-7) \begin{vmatrix} 1 & -1 \\ 1 & 1 \end{vmatrix} \\
    &= 0 + 2(2 + 3) - 7(1 + 1) \\
    &= 10 - 14 \\
    &= -4
\end{align}

Tercer determinante:
\begin{align}
    \begin{vmatrix} 0 & 2 & -7 \\ 1 & 1 & -3 \\ 1 & 0 & 2 \end{vmatrix} &= 0 - 2 \begin{vmatrix} 1 & -3 \\ 1 & 2 \end{vmatrix} + (-7) \begin{vmatrix} 1 & 1 \\ 1 & 0 \end{vmatrix} \\
    &= 0 - 2(2 + 3) - 7(0 - 1) \\
    &= -10 + 7 \\
    &= -3
\end{align}

Cuarto determinante:
\begin{align}
    \begin{vmatrix} 0 & 2 & -2 \\ 1 & 1 & -1 \\ 1 & 0 & 1 \end{vmatrix} &= 0 - 2 \begin{vmatrix} 1 & -1 \\ 1 & 1 \end{vmatrix} + (-2) \begin{vmatrix} 1 & 1 \\ 1 & 0 \end{vmatrix} \\
    &= 0 - 2(1 + 1) - 2(0 - 1) \\
    &= -4 + 2 \\
    &= -2
\end{align}

Por lo tanto:
\begin{align}
    \det(A_w) &= 1(-1) - 1(-4) + 1(-3) - 1(-2) \\
    &= -1 + 4 - 3 + 2 \\
    &= 2
\end{align}

Aplicando la regla de Cramer:
\begin{align}
x &= \frac{\det(A_x)}{\det(A)} = \frac{-2}{-2} = 1 \\
y &= \frac{\det(A_y)}{\det(A)} = \frac{4}{-2} = -2 \\
z &= \frac{\det(A_z)}{\det(A)} = \frac{-4}{-2} = 2 \\
w &= \frac{\det(A_w)}{\det(A)} = \frac{2}{-2} = -1
\end{align}

Por lo tanto, la solución del sistema es: $x = 1$, $y = -2$, $z = 2$, $w = -1$.


\subsection{Discusión}

\subsection{Conclusión}

% ========================================
% SECCIÓN 5
% ========================================
\section{Problema 5}

\subsection{Enunciado}
Mostrar si los siguientes puntos son concíclicos o no:
$$p_1 = (-1, 6), \quad p_2 = (-1, 2), \quad p_3 = (1, 4), \quad p_4 = (0, 4 - \sqrt{3})$$

\subsection{Metodología}
Para determinar si cuatro puntos son concíclicos, utilizaremos el criterio del determinante. 

Los puntos $p_1, p_2, p_3, p_4$ son concíclicos si y solo si el determinante de la matriz formada por sus coordenadas es cero:

\begin{align}
\begin{vmatrix}
x_1 & y_1 & x_1^2 + y_1^2 & 1 \\
x_2 & y_2 & x_2^2 + y_2^2 & 1 \\
x_3 & y_3 & x_3^2 + y_3^2 & 1 \\
x_4 & y_4 & x_4^2 + y_4^2 & 1
\end{vmatrix} = 0
\end{align}

Este criterio se basa en el hecho de que cuatro puntos concíclicos satisfacen la misma ecuación de círculo $x^2 + y^2 + Dx + Ey + F = 0$, lo que genera dependencia lineal en el sistema.

\subsection{Resultados}
\setcounter{equation}{0}

Primero, calculemos los valores de $x_i^2 + y_i^2$ para cada punto:

Para $p_1 = (-1, 6)$:
\begin{align}
x_1^2 + y_1^2 = (-1)^2 + 6^2 = 1 + 36 = 37
\end{align}

Para $p_2 = (-1, 2)$:
\begin{align}
x_2^2 + y_2^2 = (-1)^2 + 2^2 = 1 + 4 = 5
\end{align}

Para $p_3 = (1, 4)$:
\begin{align}
x_3^2 + y_3^2 = 1^2 + 4^2 = 1 + 16 = 17
\end{align}

Para $p_4 = (0, 4 - \sqrt{3})$:
\begin{align}
x_4^2 + y_4^2 &= 0^2 + (4 - \sqrt{3})^2 \\
&= (4 - \sqrt{3})^2 \\
&= 16 - 8\sqrt{3} + 3 \\
&= 19 - 8\sqrt{3}
\end{align}

La matriz del determinante es:
\begin{align}
\det = \begin{vmatrix}
-1 & 6 & 37 & 1 \\
-1 & 2 & 5 & 1 \\
1 & 4 & 17 & 1 \\
0 & 4-\sqrt{3} & 19-8\sqrt{3} & 1
\end{vmatrix}
\end{align}

Calculemos el determinante expandiendo por la cuarta columna (ya que todos los elementos son 1):

\begin{align}
\det &= 1 \cdot \begin{vmatrix}
-1 & 2 & 5 \\
1 & 4 & 17 \\
0 & 4-\sqrt{3} & 19-8\sqrt{3}
\end{vmatrix} - 1 \cdot \begin{vmatrix}
-1 & 6 & 37 \\
1 & 4 & 17 \\
0 & 4-\sqrt{3} & 19-8\sqrt{3}
\end{vmatrix} \\
&\quad + 1 \cdot \begin{vmatrix}
-1 & 6 & 37 \\
-1 & 2 & 5 \\
0 & 4-\sqrt{3} & 19-8\sqrt{3}
\end{vmatrix} - 1 \cdot \begin{vmatrix}
-1 & 6 & 37 \\
-1 & 2 & 5 \\
1 & 4 & 17
\end{vmatrix}
\end{align}

Calculemos cada determinante $3 \times 3$:

Primer determinante, expandiendo por la tercera fila:
\begin{align}
\begin{vmatrix}
-1 & 2 & 5 \\
1 & 4 & 17 \\
0 & 4-\sqrt{3} & 19-8\sqrt{3}
\end{vmatrix} &= 0 - (4-\sqrt{3}) \begin{vmatrix} -1 & 5 \\ 1 & 17 \end{vmatrix} + (19-8\sqrt{3}) \begin{vmatrix} -1 & 2 \\ 1 & 4 \end{vmatrix} \\
&= -(4-\sqrt{3})(-17-5) + (19-8\sqrt{3})(-4-2) \\
&= -(4-\sqrt{3})(-22) + (19-8\sqrt{3})(-6) \\
&= 22(4-\sqrt{3}) - 6(19-8\sqrt{3}) \\
&= 88 - 22\sqrt{3} - 114 + 48\sqrt{3} \\
&= -26 + 26\sqrt{3} \\
&= 26(\sqrt{3} - 1)
\end{align}

Segundo determinante, expandiendo por la tercera fila:
\begin{align}
\begin{vmatrix}
-1 & 6 & 37 \\
1 & 4 & 17 \\
0 & 4-\sqrt{3} & 19-8\sqrt{3}
\end{vmatrix} &= 0 - (4-\sqrt{3}) \begin{vmatrix} -1 & 37 \\ 1 & 17 \end{vmatrix} + (19-8\sqrt{3}) \begin{vmatrix} -1 & 6 \\ 1 & 4 \end{vmatrix} \\
&= -(4-\sqrt{3})(-17-37) + (19-8\sqrt{3})(-4-6) \\
&= -(4-\sqrt{3})(-54) + (19-8\sqrt{3})(-10) \\
&= 54(4-\sqrt{3}) - 10(19-8\sqrt{3}) \\
&= 216 - 54\sqrt{3} - 190 + 80\sqrt{3} \\
&= 26 + 26\sqrt{3} \\
&= 26(1 + \sqrt{3})
\end{align}

Tercer determinante, expandiendo por la tercera fila:
\begin{align}
\begin{vmatrix}
-1 & 6 & 37 \\
-1 & 2 & 5 \\
0 & 4-\sqrt{3} & 19-8\sqrt{3}
\end{vmatrix} &= 0 - (4-\sqrt{3}) \begin{vmatrix} -1 & 37 \\ -1 & 5 \end{vmatrix} + (19-8\sqrt{3}) \begin{vmatrix} -1 & 6 \\ -1 & 2 \end{vmatrix} \\
&= -(4-\sqrt{3})(-5+37) + (19-8\sqrt{3})(-2+6) \\
&= -(4-\sqrt{3})(32) + (19-8\sqrt{3})(4) \\
&= -32(4-\sqrt{3}) + 4(19-8\sqrt{3}) \\
&= -128 + 32\sqrt{3} + 76 - 32\sqrt{3} \\
&= -52
\end{align}

Cuarto determinante:
\begin{align}
\begin{vmatrix}
-1 & 6 & 37 \\
-1 & 2 & 5 \\
1 & 4 & 17
\end{vmatrix} &= -1 \begin{vmatrix} 2 & 5 \\ 4 & 17 \end{vmatrix} - 6 \begin{vmatrix} -1 & 5 \\ 1 & 17 \end{vmatrix} + 37 \begin{vmatrix} -1 & 2 \\ 1 & 4 \end{vmatrix} \\
&= -1(34-20) - 6(-17-5) + 37(-4-2) \\
&= -14 - 6(-22) + 37(-6) \\
&= -14 + 132 - 222 \\
&= -104
\end{align}

Sustituyendo en el determinante principal:
\begin{align}
\det &= 1 \cdot 26(\sqrt{3} - 1) - 1 \cdot 26(1 + \sqrt{3}) + 1 \cdot (-52) - 1 \cdot (-104) \\
&= 26(\sqrt{3} - 1) - 26(1 + \sqrt{3}) - 52 + 104 \\
&= 26\sqrt{3} - 26 - 26 - 26\sqrt{3} - 52 + 104 \\
&= -52 + 52 \\
&= 0
\end{align}

Por lo tanto, $\det = 0$, lo que confirma que los cuatro puntos son concíclicos.

\subsection{Discusión}

\subsection{Conclusión}

% ========================================
% SECCIÓN 6
% ========================================
\section{Problema 6}

\subsection{Enunciado}
Calcular los siguientes límites:

a) $\lim_{x \to 0} \frac{e^x - 1}{x}$

b) $\lim_{(x,y) \to (0,0)} \frac{e^{xy} - 1}{y}$

\subsection{Metodología}

\subsection{Resultados}
\setcounter{equation}{0}

\subsection{Discusión}

\subsection{Conclusión}

% ========================================
% SECCIÓN 7
% ========================================
\section{Problema 7}

\subsection{Enunciado}
Calcular $\nabla f(1, 1, 1)$ y $\nabla^2 f(1, 1, 1)$ donde $f(x, y, z) = (x + z)e^{x-y}$.

\subsection{Metodología}

\subsection{Resultados}
\setcounter{equation}{0}

\subsection{Discusión}

\subsection{Conclusión}

% ========================================
% SECCIÓN 9
% ========================================
\section{Problema 9}

\subsection{Enunciado}
Sean $\mathbf{x}$ y $\mathbf{1} := (1, 1, \ldots, 1)$ ambos en $\mathbb{R}^n$. Considerar la función de Rosenbrock:
$$f(\mathbf{x}) = \sum_{i=1}^{n-1} \left[100(x_{i+1} - x_i^2)^2 + (x_i - 1)^2\right]$$

Calcular $\nabla f(\mathbf{1})$.

\subsection{Metodología}

\subsection{Resultados}
\setcounter{equation}{0}

\subsection{Discusión}

\subsection{Conclusión}

\end{document}