\documentclass{article}
\usepackage{graphicx}
\usepackage{amsmath}
\usepackage{amssymb}
\usepackage{amsfonts}

\title{MEMAD-T01}
\author{ALEJANDRO ZARATE MACIAS}
\date{25 de Agosto 2025}

\begin{document}

\maketitle

% ========================================
% INTRODUCCIÓN
% ========================================
\section*{Introducción}

Para esta tarea se busca la resolución de problemas relacionados con álgebra vectorial, cálculo multivariable y métodos de optimización.
La idea es utilizar lo aprendido en los videos proporcionados como material de estudio, asi como los libros sugeridos para el curso. Todo esto con el fin de aplicar:
\begin{itemize}
    \item Funciones
    \item Matrices
    \item Normas
    \item Límites
    \item Derivación
    \item Gradientes
    \item Entre otros.
\end{itemize}

% ========================================
% SECCIÓN 1
% ========================================
\section{Problema 1}

\subsection{Enunciado}
Encontrar los ángulos (ambos) entre los vectores $\vec{p} = (-1, 3, 1)$ y $\vec{q} = (-2, -3, -7)$.

\subsection{Metodología}
Para la resolución de este problema, se debe utilizar la fórmula

\begin{align}
\cos(\theta) = \frac{<\vec{p},\vec{q}>}{||\vec{p}|| ||\vec{q}||}
\end{align}

donde $<\vec{p},\vec{q}>$ es el producto punto de los vectores y $||\vec{p}||$ y $||\vec{q}||$ son las normas de los vectores.
Para esto, se deben seguir los siguientes pasos:
\begin{enumerate}
    \item[-]  Calcular el producto punto $<\vec{p},\vec{q}>$.
    \item[-] Calcular las normas $||\vec{p}||$ y $||\vec{q}||$.
    \item[-] Sustituir los valores en la fórmula.
    \item[-] Despejar $\theta$.
    \item[-] Calcular los ángulos.
\end{enumerate}

\subsection{Resultados}
\setcounter{equation}{0}

Siguiendo los pasos mencionados, en la metodología, tenemos que el producto punto es:
\begin{align}
<\vec{p},\vec{q}> &= (-1)(-2) + (3)(-3) + (1)(-7) \\
&= 2 - 9 - 7 \\
&= -14
\end{align}

Ahora calculamos las normas.
Para $\vec{p}$:
\begin{align}
||\vec{p}|| &= \sqrt{(-1)^2 + 3^2 + 1^2} \\
&= \sqrt{1 + 9 + 1} \\
&= \sqrt{11}
\end{align}

Y para $\vec{q}$:
\begin{align}
||\vec{q}|| &= \sqrt{(-2)^2 + (-3)^2 + (-7)^2} \\
&= \sqrt{4 + 9 + 49} \\
&= \sqrt{62}
\end{align}

Sustituyendo los valores en la fórmula, tenemos:
\begin{align}
\cos(\theta) &= \frac{-14}{(\sqrt{11}) (\sqrt{62})} \\
&= \frac{-14}{\sqrt{682}} 
\end{align}

Despejando $\theta$, tenemos:
\begin{align}
\theta = \arccos(\frac{-14}{\sqrt{682}}) 
\end{align}

Por lo tanto:
\begin{align}
\theta_1 &= \arccos(-0.536) \\
&\approx 122.41^\circ
\end{align}

Como resultado dio mayor a $90\circ$, este es el ángulo obtuso entre los vectores.

El ángulo agudo correspondiente es:
\begin{align}
\theta_2 &= 180^\circ - 122.41^\circ \\
&= 57.59^\circ
\end{align}

\subsection{Discusión}
Los resultados obtenidos confirman que los vectores dados forman un ángulo obtuso de aproximadamente 122.41°, lo cual es consistente con el hecho de que su producto punto es negativo (-14). La metodología empleada, basada en la fórmula del coseno del ángulo entre vectores, es directa y permite obtener ambos ángulos posibles entre las direcciones de los vectores.

\subsection{Conclusión}
Se determinaron exitosamente los dos ángulos entre los vectores $\vec{p} = (-1, 3, 1)$ y $\vec{q} = (-2, -3, -7)$: el ángulo obtuso de 122.41° y el ángulo agudo de 57.59°. El cálculo se realizó aplicando correctamente la fórmula del producto punto y las propiedades de las normas vectoriales.

% ========================================
% SECCIÓN 2
% ========================================
\section{Problema 2}

\subsection{Enunciado}
Encontrar dos vectores no paralelos que sean ortogonales a $\vec{p} = (1, 1, 1)$.

\subsection{Metodología}

Para encontrar vectores ortogonales a $\vec{p} = (1, 1, 1)$, debemos encontrar vectores $\vec{v}$ tales que el producto punto $\vec{p} \cdot \vec{v} = 0$.

Si $\vec{v} = (x, y, z)$, entonces:
\begin{align}
\vec{p} \cdot \vec{v} = (1, 1, 1) \cdot (x, y, z) = x + y + z = 0
\end{align}

Para verificar que los vectores no sean paralelos, utilizaremos el concepto de proporcionalidad: dos vectores $\vec{v_1} = (a_1, b_1, c_1)$ y $\vec{v_2} = (a_2, b_2, c_2)$ son paralelos si y solo si las razones entre sus componentes correspondientes son iguales, es decir:
$$\frac{a_1}{a_2} = \frac{b_1}{b_2} = \frac{c_1}{c_2}$$

Entonces, para encontrar dos vectores no paralelos que sean ortogonales a $\vec{p}$, seguiremos estos pasos:
\begin{enumerate}
    \item[-] Elegir vectores que cumplan la condición $x + y + z = 0$.
    \item[-] Verificar que ambos sean ortogonales a $\vec{p}$.
    \item[-] Verificar que no sean paralelos usando el concepto de proporcionalidad.
\end{enumerate}

\subsection{Resultados}
\setcounter{equation}{0}

Vector 1: $\vec{v_1} = (5,-1,-4)$ \\
Verificación de ortogonalidad: \\
\begin{align}
\vec{p} \cdot \vec{v_1} &= (1, 1, 1) \cdot (5, -1, -4) \\
&= 1(5) + 1(-1) + 1(-4) \\
&= 5 - 1 - 4 \\
&= 0
\end{align}

Vector 2: $\vec{v_2} = (1, 2, -3)$ \\
Verificación de ortogonalidad: \\
\begin{align}
\vec{p} \cdot \vec{v_2} &= (1, 1, 1) \cdot (1, 2, -3) \\
&= 1(1) + 1(2) + 1(-3) \\
&= 1 + 2 - 3 \\
&= 0
\end{align}

Ahora debemos asegurarnos de que los vectores no sean paralelos, para esto usamos proporcionalidad:
\begin{align}
\frac{5}{1} &= 5 \\
\frac{-1}{2} &= -0.5 \\
\frac{-4}{-3} &\approx 1.33
\end{align}

Como $5 \neq -0.5 \neq 1.33$, los vectores no son paralelos.

\subsection{Discusión}
La metodología empleada demuestra que es posible encontrar infinitos vectores ortogonales a un vector dado en $\mathbb{R}^3$, ya que la condición $x + y + z = 0$ define un plano que pasa por el origen. La verificación de no paralelismo mediante el análisis de proporcionalidad es un método bastante sencillo pero eficaz en este tipo de casos, ya que de otra manera, tendríamos que calcular el producto cruzado o ángulos entre los dos vectores, lo cual conlleva a un incremento en los pasos para conocer este dato.

\subsection{Conclusión}
Se encontraron exitosamente dos vectores no paralelos ortogonales a $\vec{p} = (1, 1, 1)$: $\vec{v_1} = (5, -1, -4)$ y $\vec{v_2} = (1, 2, -3)$. Ambos vectores cumplen la condición de ortogonalidad y se verificó que no son linealmente dependientes entre sí.

% ========================================
% SECCIÓN 3
% ========================================
\section{Problema 3}

\subsection{Enunciado}
Sea $n$ un número natural y $A$ como sigue:
$$A = \begin{vmatrix}
n & n+1 & n+2 \\
n+3 & n+4 & n+5 \\
n+6 & n+7 & n+8
\end{vmatrix}$$

Mostrar que $\det(A)$ permanece constante con respecto a $n$.

\subsection{Metodología}

Para demostrar que $\det(A)$ permanece constante con respecto a $n$, calcularemos el determinante para un valor específico de $n$ y luego analizaremos la estructura de la matriz para mostrar dependencia lineal entre las filas.

\subsection{Resultados}
\setcounter{equation}{0}

Calculemos el determinante para $n = 1$:
$$A = \begin{vmatrix}
1 & 2 & 3 \\
4 & 5 & 6 \\
7 & 8 & 9
\end{vmatrix}$$

Expandiendo por la primera fila:
\begin{align}
\det(A) &= 1 \begin{vmatrix} 5 & 6 \\ 8 & 9 \end{vmatrix} - 2 \begin{vmatrix} 4 & 6 \\ 7 & 9 \end{vmatrix} + 3 \begin{vmatrix} 4 & 5 \\ 7 & 8 \end{vmatrix} \\
&= 1 (5 \times 9 - 6 \times 8) - 2 (4 \times 9 - 6 \times 7) + 3 (4 \times 8 - 5 \times 7) \\
&= 1 (45 - 48) - 2 (36 - 42) + 3 (32 - 35) \\
&= 1 (-3) - 2 (-6) + 3 (-3) \\
&= -3 + 12 - 9 \\
&= 0
\end{align}

El hecho de que $\det(A) = 0$ nos indica una posible dependencia lineal entre las filas.

Analicemos la estructura de la matriz. Observemos que cada fila es una progresión aritmética con diferencia común 1, y entre filas consecutivas hay una diferencia común de 3 en cada posición:

Para la matriz general:
$$\begin{pmatrix}
n & n+1 & n+2 \\
n+3 & n+4 & n+5 \\
n+6 & n+7 & n+8
\end{pmatrix}$$

Notemos que la tercera fila menos la segunda fila es igual a la segunda fila menos la primera fila:
\begin{align}
\text{Fila 3} - \text{Fila 2} &= (3, 3, 3) \\
\text{Fila 2} - \text{Fila 1} &= (3, 3, 3)
\end{align}

Por lo tanto: $\text{Fila 1} - 2 \times \text{Fila 2} + \text{Fila 3} = (0, 0, 0)$

Esta relación lineal entre las filas demuestra que son linealmente dependientes, lo que implica que $\det(A) = 0$ para cualquier valor de $n$.

\subsection{Discusión}
El resultado obtenido nos muestra que antes de intentar resolver estos problemas mediante formulas o métodos mas generales, podemos recurrir a métodos mas analíticos. El conocer cierta propiedades sobre las matrices y determinantes hace mas sencillo resolver este tipo de problemas.

\subsection{Conclusión}
Se demostró analíticamente que $\det(A) = 0$ para cualquier valor natural de $n$, confirmando que la matriz permanece singular independientemente del parámetro.

% ========================================
% SECCIÓN 4
% ========================================
\section{Problema 4}

\subsection{Enunciado}
Resolver el siguiente sistema de ecuaciones lineales usando determinantes:
\begin{align}
x + y + z &= 1 \\
2y - 2z - w &= -7 \\
x + y - z &= -3 \\
x + z + w &= 2
\end{align}

\subsection{Metodología}

Para resolver el sistema de ecuaciones lineales usando determinantes, utilizaremos el método de Cramer. El sistema se puede escribir en forma matricial como $A\mathbf{x} = \mathbf{b}$, donde:

\begin{align*}
    A = \begin{pmatrix}
    1 & 1 & 1 & 0 \\
    0 & 2 & -2 & -1 \\
    1 & 1 & -1 & 0 \\
    1 & 0 & 1 & 1
    \end{pmatrix}, \quad \mathbf{x} = \begin{pmatrix} x \\ y \\ z \\ w \end{pmatrix}, \quad \mathbf{b} = \begin{pmatrix} 1 \\ -7 \\ -3 \\ 2 \end{pmatrix}
\end{align*}

El método de Cramer establece que:

\begin{align*}
    x = \frac{\det(A_x)}{\det(A)}, \quad y = \frac{\det(A_y)}{\det(A)}, \quad z = \frac{\det(A_z)}{\det(A)}, \quad w = \frac{\det(A_w)}{\det(A)}    
\end{align*}

donde $A_x$, $A_y$, $A_z$, $A_w$ son las matrices obtenidas al reemplazar la columna correspondiente de $A$ por el vector $\mathbf{b}$.

Los pasos a seguir son:
\begin{enumerate}
    \item[-] Calcular $\det(A)$.
    \item[-] Calcular $\det(A_x)$, $\det(A_y)$, $\det(A_z)$, $\det(A_w)$.
    \item[-] Aplicar el método de Cramer para obtener cada variable.
\end{enumerate}

\subsection{Resultados}
\setcounter{equation}{0}

Primero calculamos $\det(A)$ expandiendo por la primera fila:

\begin{align}
    det(A) = \begin{vmatrix}
    1 & 1 & 1 & 0 \\
    0 & 2 & -2 & -1 \\
    1 & 1 & -1 & 0 \\
    1 & 0 & 1 & 1
    \end{vmatrix}
\end{align}

Quedando de la siguiente manera:
\begin{align}
    det(A) = 1 \cdot \begin{vmatrix}
    2 & -2 & -1 \\
    1 & -1 & 0 \\
    0 & 1 & 1
    \end{vmatrix} - 1 \cdot \begin{vmatrix}
    0 & -2 & -1 \\
    1 & -1 & 0 \\
    1 & 1 & 1
    \end{vmatrix} + 1 \cdot \begin{vmatrix}
    0 & 2 & -1 \\
    1 & 1 & 0 \\
    1 & 0 & 1
    \end{vmatrix}
\end{align}

En este caso, el cuarto elemento de la primera fila fue omitido, ya que al ser $0$, su multiplicación dará $0$. Además, podemos notar que los demás elementos de la primera fila son $1$, por lo que podemos simplificar la expresión a: \\

\begin{align}
    det(A) = \begin{vmatrix}
    2 & -2 & -1 \\
    1 & -1 & 0 \\
    0 & 1 & 1
    \end{vmatrix} - \begin{vmatrix}
    0 & -2 & -1 \\
    1 & -1 & 0 \\
    1 & 1 & 1
    \end{vmatrix} + \begin{vmatrix}
    0 & 2 & -1 \\
    1 & 1 & 0 \\
    1 & 0 & 1
    \end{vmatrix}
\end{align}

Para calcular los 3 determinantes de las matrices resultantes, expandiremos nuevamente por la primer fila de cada matriz.

Primer determinante:
\begin{align}
    \begin{vmatrix}
    2 & -2 & -1 \\
    1 & -1 & 0 \\
    0 & 1 & 1
    \end{vmatrix} &= 2 \begin{vmatrix} -1 & 0 \\ 1 & 1 \end{vmatrix} - (-2) \begin{vmatrix} 1 & 0 \\ 0 & 1 \end{vmatrix} + (-1) \begin{vmatrix} 1 & -1 \\ 0 & 1 \end{vmatrix} \\
    &= 2(-1) + 2(1) - 1(1) \\
    &= -2 + 2 - 1 \\
    &= -1
\end{align}

Segundo determinante:
\begin{align}
    \begin{vmatrix}
    0 & -2 & -1 \\
    1 & -1 & 0 \\
    1 & 1 & 1
    \end{vmatrix} &= 0 - (-2) \begin{vmatrix} 1 & 0 \\ 1 & 1 \end{vmatrix} + (-1) \begin{vmatrix} 1 & -1 \\ 1 & 1 \end{vmatrix} \\
    &= 0 + 2(1) - 1(2) \\
    &= 2 - 2 \\
    &= 0
\end{align}

Tercer determinante:
\begin{align}
    \begin{vmatrix}
    0 & 2 & -1 \\
    1 & 1 & 0 \\
    1 & 0 & 1
    \end{vmatrix} &= 0 - 2 \begin{vmatrix} 1 & 0 \\ 1 & 1 \end{vmatrix} + (-1) \begin{vmatrix} 1 & 1 \\ 1 & 0 \end{vmatrix} \\
    &= 0 - 2(1) - 1(-1) \\
    &= -2 + 1 \\
    &= -1
\end{align}

Por lo tanto:
\begin{align}
    \det(A) &= (-1) - (0) + (-1) \\
    &= -2
\end{align}

Ahora que ya conocemos el determinante de la matriz $A$, podemos calcular los determinantes de las matrices $A_x$, $A_y$, $A_z$ y $A_w$.

Calculamos $\det(A_x)$:

\begin{align}
    A_x = \begin{pmatrix}
    1 & 1 & 1 & 0 \\
    -7 & 2 & -2 & -1 \\
    -3 & 1 & -1 & 0 \\
    2 & 0 & 1 & 1
    \end{pmatrix}
\end{align}

Expandiendo por la primera fila:
\begin{align}
    \det(A_x) &= 1 \cdot \begin{vmatrix} 2 & -2 & -1 \\ 1 & -1 & 0 \\ 0 & 1 & 1 \end{vmatrix} - 1 \cdot \begin{vmatrix} -7 & -2 & -1 \\ -3 & -1 & 0 \\ 2 & 1 & 1 \end{vmatrix} + 1 \cdot \begin{vmatrix} -7 & 2 & -1 \\ -3 & 1 & 0 \\ 2 & 0 & 1 \end{vmatrix}
\end{align}

Calculamos cada determinante 3×3:

Primer determinante:
\begin{align}
    \begin{vmatrix} 2 & -2 & -1 \\ 1 & -1 & 0 \\ 0 & 1 & 1 \end{vmatrix} &= 2 \begin{vmatrix} -1 & 0 \\ 1 & 1 \end{vmatrix} - (-2) \begin{vmatrix} 1 & 0 \\ 0 & 1 \end{vmatrix} + (-1) \begin{vmatrix} 1 & -1 \\ 0 & 1 \end{vmatrix} \\
    &= 2(-1) + 2(1) - 1(1) \\
    &= -2 + 2 - 1 \\
    &= -1
\end{align}

Segundo determinante:
\begin{align}
    \begin{vmatrix} -7 & -2 & -1 \\ -3 & -1 & 0 \\ 2 & 1 & 1 \end{vmatrix} &= -7 \begin{vmatrix} -1 & 0 \\ 1 & 1 \end{vmatrix} - (-2) \begin{vmatrix} -3 & 0 \\ 2 & 1 \end{vmatrix} + (-1) \begin{vmatrix} -3 & -1 \\ 2 & 1 \end{vmatrix} \\
    &= -7(-1) + 2(-3) - 1(-3 + 2) \\
    &= 7 - 6 - 1(-1) \\
    &= 7 - 6 + 1 \\
    &= 2
\end{align}

Tercer determinante:
\begin{align}
    \begin{vmatrix} -7 & 2 & -1 \\ -3 & 1 & 0 \\ 2 & 0 & 1 \end{vmatrix} &= -7 \begin{vmatrix} 1 & 0 \\ 0 & 1 \end{vmatrix} - 2 \begin{vmatrix} -3 & 0 \\ 2 & 1 \end{vmatrix} + (-1) \begin{vmatrix} -3 & 1 \\ 2 & 0 \end{vmatrix} \\
    &= -7(1) - 2(-3) - 1(0 - 2) \\
    &= -7 + 6 - 1(-2) \\
    &= -7 + 6 + 2 \\
    &= 1
\end{align}

Por lo tanto:
\begin{align}
    \det(A_x) &= 1(-1) - 1(2) + 1(1) - 0 \\
    &= -1 - 2 + 1 \\
    &= -2
\end{align}

Calculamos $\det(A_y)$:

\begin{align}
    A_y = \begin{pmatrix}
    1 & 1 & 1 & 0 \\
    0 & -7 & -2 & -1 \\
    1 & -3 & -1 & 0 \\
    1 & 2 & 1 & 1
    \end{pmatrix}
\end{align}

Expandiendo por la primera fila:
\begin{align}
    \det(A_y) &= 1 \cdot \begin{vmatrix} -7 & -2 & -1 \\ -3 & -1 & 0 \\ 2 & 1 & 1 \end{vmatrix} - 1 \cdot \begin{vmatrix} 0 & -2 & -1 \\ 1 & -1 & 0 \\ 1 & 1 & 1 \end{vmatrix} + 1 \cdot \begin{vmatrix} 0 & -7 & -1 \\ 1 & -3 & 0 \\ 1 & 2 & 1 \end{vmatrix}
\end{align}

Calculamos cada determinante 3×3:

Primer determinante:
\begin{align}
    \begin{vmatrix} -7 & -2 & -1 \\ -3 & -1 & 0 \\ 2 & 1 & 1 \end{vmatrix} &= -7 \begin{vmatrix} -1 & 0 \\ 1 & 1 \end{vmatrix} - (-2) \begin{vmatrix} -3 & 0 \\ 2 & 1 \end{vmatrix} + (-1) \begin{vmatrix} -3 & -1 \\ 2 & 1 \end{vmatrix} \\
    &= -7(-1) + 2(-3) - 1(-3 + 2) \\
    &= 7 - 6 + 1 \\
    &= 2
\end{align}

Segundo determinante:
\begin{align}
    \begin{vmatrix} 0 & -2 & -1 \\ 1 & -1 & 0 \\ 1 & 1 & 1 \end{vmatrix} &= 0 - (-2) \begin{vmatrix} 1 & 0 \\ 1 & 1 \end{vmatrix} + (-1) \begin{vmatrix} 1 & -1 \\ 1 & 1 \end{vmatrix} \\
    &= 0 + 2(1) - 1(1 + 1) \\
    &= 2 - 2 \\
    &= 0
\end{align}

Tercer determinante:
\begin{align}
    \begin{vmatrix} 0 & -7 & -1 \\ 1 & -3 & 0 \\ 1 & 2 & 1 \end{vmatrix} &= 0 - (-7) \begin{vmatrix} 1 & 0 \\ 1 & 1 \end{vmatrix} + (-1) \begin{vmatrix} 1 & -3 \\ 1 & 2 \end{vmatrix} \\
    &= 0 + 7(1) - 1(2 + 3) \\
    &= 7 - 5 \\
    &= 2
\end{align}

Por lo tanto:
\begin{align}
    \det(A_y) &= 1(2) - 1(0) + 1(2) \\
    &= 2 + 0 + 2 \\
    &= 4
\end{align}

Calculamos $\det(A_z)$:

\begin{align}
    A_z = \begin{pmatrix}
    1 & 1 & 1 & 0 \\
    0 & 2 & -7 & -1 \\
    1 & 1 & -3 & 0 \\
    1 & 0 & 2 & 1
    \end{pmatrix}
\end{align}

Expandiendo por la primera fila:
\begin{align}
    \det(A_z) &= 1 \cdot \begin{vmatrix} 2 & -7 & -1 \\ 1 & -3 & 0 \\ 0 & 2 & 1 \end{vmatrix} - 1 \cdot \begin{vmatrix} 0 & -7 & -1 \\ 1 & -3 & 0 \\ 1 & 2 & 1 \end{vmatrix} + 1 \cdot \begin{vmatrix} 0 & 2 & -1 \\ 1 & 1 & 0 \\ 1 & 0 & 1 \end{vmatrix}
\end{align}

Calculamos cada determinante 3×3:

Primer determinante:
\begin{align}
    \begin{vmatrix} 2 & -7 & -1 \\ 1 & -3 & 0 \\ 0 & 2 & 1 \end{vmatrix} &= 2 \begin{vmatrix} -3 & 0 \\ 2 & 1 \end{vmatrix} - (-7) \begin{vmatrix} 1 & 0 \\ 0 & 1 \end{vmatrix} + (-1) \begin{vmatrix} 1 & -3 \\ 0 & 2 \end{vmatrix} \\
    &= 2(-3) + 7(1) - 1(2) \\
    &= -6 + 7 - 2 \\
    &= -1
\end{align}

Segundo determinante:
\begin{align}
    \begin{vmatrix} 0 & -7 & -1 \\ 1 & -3 & 0 \\ 1 & 2 & 1 \end{vmatrix} &= 0 - (-7) \begin{vmatrix} 1 & 0 \\ 1 & 1 \end{vmatrix} + (-1) \begin{vmatrix} 1 & -3 \\ 1 & 2 \end{vmatrix} \\
    &= 0 + 7(1) - 1(2 + 3) \\
    &= 7 - 5 \\
    &= 2
\end{align}

Tercer determinante:
\begin{align}
    \begin{vmatrix} 0 & 2 & -1 \\ 1 & 1 & 0 \\ 1 & 0 & 1 \end{vmatrix} &= 0 - 2 \begin{vmatrix} 1 & 0 \\ 1 & 1 \end{vmatrix} + (-1) \begin{vmatrix} 1 & 1 \\ 1 & 0 \end{vmatrix} \\
    &= 0 - 2(1) - 1(0 - 1) \\
    &= -2 + 1 \\
    &= -1
\end{align}

Por lo tanto:
\begin{align}
    \det(A_z) &= 1(-1) - 1(2) + 1(-1) \\
    &= -1 - 2 - 1 \\
    &= -4
\end{align}

Calculamos $\det(A_w)$:

\begin{align}
    A_w = \begin{pmatrix}
    1 & 1 & 1 & 1 \\
    0 & 2 & -2 & -7 \\
    1 & 1 & -1 & -3 \\
    1 & 0 & 1 & 2
    \end{pmatrix}
\end{align}

Expandiendo por la primera fila:
\begin{align}
    \det(A_w) &= 1 \cdot \begin{vmatrix} 2 & -2 & -7 \\ 1 & -1 & -3 \\ 0 & 1 & 2 \end{vmatrix} - 1 \cdot \begin{vmatrix} 0 & -2 & -7 \\ 1 & -1 & -3 \\ 1 & 1 & 2 \end{vmatrix} + 1 \cdot \begin{vmatrix} 0 & 2 & -7 \\ 1 & 1 & -3 \\ 1 & 0 & 2 \end{vmatrix} - 1 \cdot \begin{vmatrix} 0 & 2 & -2 \\ 1 & 1 & -1 \\ 1 & 0 & 1 \end{vmatrix}
\end{align}

Calculamos cada determinante 3×3:

Primer determinante:
\begin{align}
    \begin{vmatrix} 2 & -2 & -7 \\ 1 & -1 & -3 \\ 0 & 1 & 2 \end{vmatrix} &= 2 \begin{vmatrix} -1 & -3 \\ 1 & 2 \end{vmatrix} - (-2) \begin{vmatrix} 1 & -3 \\ 0 & 2 \end{vmatrix} + (-7) \begin{vmatrix} 1 & -1 \\ 0 & 1 \end{vmatrix} \\
    &= 2(-2 + 3) + 2(2) - 7(1) \\
    &= 2 + 4 - 7 \\
    &= -1
\end{align}

Segundo determinante:
\begin{align}
    \begin{vmatrix} 0 & -2 & -7 \\ 1 & -1 & -3 \\ 1 & 1 & 2 \end{vmatrix} &= 0 - (-2) \begin{vmatrix} 1 & -3 \\ 1 & 2 \end{vmatrix} + (-7) \begin{vmatrix} 1 & -1 \\ 1 & 1 \end{vmatrix} \\
    &= 0 + 2(2 + 3) - 7(1 + 1) \\
    &= 10 - 14 \\
    &= -4
\end{align}

Tercer determinante:
\begin{align}
    \begin{vmatrix} 0 & 2 & -7 \\ 1 & 1 & -3 \\ 1 & 0 & 2 \end{vmatrix} &= 0 - 2 \begin{vmatrix} 1 & -3 \\ 1 & 2 \end{vmatrix} + (-7) \begin{vmatrix} 1 & 1 \\ 1 & 0 \end{vmatrix} \\
    &= 0 - 2(2 + 3) - 7(0 - 1) \\
    &= -10 + 7 \\
    &= -3
\end{align}

Cuarto determinante:
\begin{align}
    \begin{vmatrix} 0 & 2 & -2 \\ 1 & 1 & -1 \\ 1 & 0 & 1 \end{vmatrix} &= 0 - 2 \begin{vmatrix} 1 & -1 \\ 1 & 1 \end{vmatrix} + (-2) \begin{vmatrix} 1 & 1 \\ 1 & 0 \end{vmatrix} \\
    &= 0 - 2(1 + 1) - 2(0 - 1) \\
    &= -4 + 2 \\
    &= -2
\end{align}

Por lo tanto:
\begin{align}
    \det(A_w) &= 1(-1) - 1(-4) + 1(-3) - 1(-2) \\
    &= -1 + 4 - 3 + 2 \\
    &= 2
\end{align}

Aplicando la regla de Cramer:
\begin{align}
x &= \frac{\det(A_x)}{\det(A)} = \frac{-2}{-2} = 1 \\
y &= \frac{\det(A_y)}{\det(A)} = \frac{4}{-2} = -2 \\
z &= \frac{\det(A_z)}{\det(A)} = \frac{-4}{-2} = 2 \\
w &= \frac{\det(A_w)}{\det(A)} = \frac{2}{-2} = -1
\end{align}

Por lo tanto, la solución del sistema es: $x = 1$, $y = -2$, $z = 2$, $w = -1$.


\subsection{Discusión}
El método de Cramer proporcionó una solución exacta para el sistema de ecuaciones lineales. No obstante, cuando nos enfrentamos a matrices de gran tamaño, se vuelve cada vez más compleja la obtención de un resultado, puesto que hay que calcular el determinante general sumado a los determinantes por cada incógnita. Para esto existen métodos más simples como Gauss o Gauss-Jordan con los que en una menor cantidad de pasos, podemos llegar al resultado.

\subsection{Conclusión}
Se resolvió exitosamente el sistema de ecuaciones lineales usando el método de Cramer, obteniendo la solución única: $x = 1$, $y = -2$, $z = 2$, $w = -1$. La metodología empleada demostró ser eficiente para sistemas de esta dimensión, proporcionando resultados exactos mediante el cálculo de determinantes.

% ========================================
% SECCIÓN 5
% ========================================
\section{Problema 5}

\subsection{Enunciado}
Mostrar si los siguientes puntos son concíclicos o no:
$$p_1 = (-1, 6), \quad p_2 = (-1, 2), \quad p_3 = (1, 4), \quad p_4 = (0, 4 - \sqrt{3})$$

\subsection{Metodología}
Para determinar si cuatro puntos son concíclicos, utilizaremos el criterio del determinante. 

Los puntos $p_1, p_2, p_3, p_4$ son concíclicos si y solo si el determinante de la matriz formada por sus coordenadas es cero:

\begin{align}
\begin{vmatrix}
x_1 & y_1 & x_1^2 + y_1^2 & 1 \\
x_2 & y_2 & x_2^2 + y_2^2 & 1 \\
x_3 & y_3 & x_3^2 + y_3^2 & 1 \\
x_4 & y_4 & x_4^2 + y_4^2 & 1
\end{vmatrix} = 0
\end{align}

Este criterio se basa en el hecho de que cuatro puntos concíclicos satisfacen la misma ecuación de círculo $x^2 + y^2 + Dx + Ey + F = 0$, lo que genera dependencia lineal en el sistema.

\subsection{Resultados}
\setcounter{equation}{0}

Primero, calculemos los valores de $x_i^2 + y_i^2$ para cada punto:

Para $p_1 = (-1, 6)$:
\begin{align}
x_1^2 + y_1^2 = (-1)^2 + 6^2 = 1 + 36 = 37
\end{align}

Para $p_2 = (-1, 2)$:
\begin{align}
x_2^2 + y_2^2 = (-1)^2 + 2^2 = 1 + 4 = 5
\end{align}

Para $p_3 = (1, 4)$:
\begin{align}
x_3^2 + y_3^2 = 1^2 + 4^2 = 1 + 16 = 17
\end{align}

Para $p_4 = (0, 4 - \sqrt{3})$:
\begin{align}
x_4^2 + y_4^2 &= 0^2 + (4 - \sqrt{3})^2 \\
&= (4 - \sqrt{3})^2 \\
&= 16 - 8\sqrt{3} + 3 \\
&= 19 - 8\sqrt{3}
\end{align}

La matriz del determinante es:
\begin{align}
\det = \begin{vmatrix}
-1 & 6 & 37 & 1 \\
-1 & 2 & 5 & 1 \\
1 & 4 & 17 & 1 \\
0 & 4-\sqrt{3} & 19-8\sqrt{3} & 1
\end{vmatrix}
\end{align}

Calculemos el determinante expandiendo por la cuarta columna (ya que todos los elementos son 1):

\begin{align}
\det &= 1 \cdot \begin{vmatrix}
-1 & 2 & 5 \\
1 & 4 & 17 \\
0 & 4-\sqrt{3} & 19-8\sqrt{3}
\end{vmatrix} - 1 \cdot \begin{vmatrix}
-1 & 6 & 37 \\
1 & 4 & 17 \\
0 & 4-\sqrt{3} & 19-8\sqrt{3}
\end{vmatrix} \\
&\quad + 1 \cdot \begin{vmatrix}
-1 & 6 & 37 \\
-1 & 2 & 5 \\
0 & 4-\sqrt{3} & 19-8\sqrt{3}
\end{vmatrix} - 1 \cdot \begin{vmatrix}
-1 & 6 & 37 \\
-1 & 2 & 5 \\
1 & 4 & 17
\end{vmatrix}
\end{align}

Calculemos cada determinante $3 \times 3$:

Primer determinante, expandiendo por la tercera fila:
\begin{align}
\begin{vmatrix}
-1 & 2 & 5 \\
1 & 4 & 17 \\
0 & 4-\sqrt{3} & 19-8\sqrt{3}
\end{vmatrix} &= 0 - (4-\sqrt{3}) \begin{vmatrix} -1 & 5 \\ 1 & 17 \end{vmatrix} + (19-8\sqrt{3}) \begin{vmatrix} -1 & 2 \\ 1 & 4 \end{vmatrix} \\
&= -(4-\sqrt{3})(-17-5) + (19-8\sqrt{3})(-4-2) \\
&= -(4-\sqrt{3})(-22) + (19-8\sqrt{3})(-6) \\
&= 22(4-\sqrt{3}) - 6(19-8\sqrt{3}) \\
&= 88 - 22\sqrt{3} - 114 + 48\sqrt{3} \\
&= -26 + 26\sqrt{3} \\
&= 26(\sqrt{3} - 1)
\end{align}

Segundo determinante, expandiendo por la tercera fila:
\begin{align}
\begin{vmatrix}
-1 & 6 & 37 \\
1 & 4 & 17 \\
0 & 4-\sqrt{3} & 19-8\sqrt{3}
\end{vmatrix} &= 0 - (4-\sqrt{3}) \begin{vmatrix} -1 & 37 \\ 1 & 17 \end{vmatrix} + (19-8\sqrt{3}) \begin{vmatrix} -1 & 6 \\ 1 & 4 \end{vmatrix} \\
&= -(4-\sqrt{3})(-17-37) + (19-8\sqrt{3})(-4-6) \\
&= -(4-\sqrt{3})(-54) + (19-8\sqrt{3})(-10) \\
&= 54(4-\sqrt{3}) - 10(19-8\sqrt{3}) \\
&= 216 - 54\sqrt{3} - 190 + 80\sqrt{3} \\
&= 26 + 26\sqrt{3} \\
&= 26(1 + \sqrt{3})
\end{align}

Tercer determinante, expandiendo por la tercera fila:
\begin{align}
\begin{vmatrix}
-1 & 6 & 37 \\
-1 & 2 & 5 \\
0 & 4-\sqrt{3} & 19-8\sqrt{3}
\end{vmatrix} &= 0 - (4-\sqrt{3}) \begin{vmatrix} -1 & 37 \\ -1 & 5 \end{vmatrix} + (19-8\sqrt{3}) \begin{vmatrix} -1 & 6 \\ -1 & 2 \end{vmatrix} \\
&= -(4-\sqrt{3})(-5+37) + (19-8\sqrt{3})(-2+6) \\
&= -(4-\sqrt{3})(32) + (19-8\sqrt{3})(4) \\
&= -32(4-\sqrt{3}) + 4(19-8\sqrt{3}) \\
&= -128 + 32\sqrt{3} + 76 - 32\sqrt{3} \\
&= -52
\end{align}

Cuarto determinante:
\begin{align}
\begin{vmatrix}
-1 & 6 & 37 \\
-1 & 2 & 5 \\
1 & 4 & 17
\end{vmatrix} &= -1 \begin{vmatrix} 2 & 5 \\ 4 & 17 \end{vmatrix} - 6 \begin{vmatrix} -1 & 5 \\ 1 & 17 \end{vmatrix} + 37 \begin{vmatrix} -1 & 2 \\ 1 & 4 \end{vmatrix} \\
&= -1(34-20) - 6(-17-5) + 37(-4-2) \\
&= -14 - 6(-22) + 37(-6) \\
&= -14 + 132 - 222 \\
&= -104
\end{align}

Sustituyendo en el determinante principal:
\begin{align}
\det &= 1 \cdot 26(\sqrt{3} - 1) - 1 \cdot 26(1 + \sqrt{3}) + 1 \cdot (-52) - 1 \cdot (-104) \\
&= 26(\sqrt{3} - 1) - 26(1 + \sqrt{3}) - 52 + 104 \\
&= 26\sqrt{3} - 26 - 26 - 26\sqrt{3} - 52 + 104 \\
&= -52 + 52 \\
&= 0
\end{align}

Por lo tanto, $\det = 0$, lo que confirma que los cuatro puntos son concíclicos.

\subsection{Discusión}
El criterio del determinante para verificar concicularidad de puntos en el plano es una herramienta bastante útil. El resultado obtenido ($\det = 0$) confirma que existe una única circunferencia que pasa por los cuatro puntos dados. Este método es particularmente útil cuando los cálculos geométricos directos serían más complejos, especialmente con puntos que involucran expresiones irracionales como $\sqrt{3}$.

\subsection{Conclusión}
Se demostró analíticamente que los puntos $p_1 = (-1, 6)$, $p_2 = (-1, 2)$, $p_3 = (1, 4)$ y $p_4 = (0, 4 - \sqrt{3})$ son concíclicos. La aplicación del criterio del determinante proporcionó una verificación rigurosa de esta propiedad geométrica mediante métodos algebraicos.

% ========================================
% SECCIÓN 6
% ========================================
\section{Problema 6}

\subsection{Enunciado}
Calcular los siguientes límites:

a) $\lim_{x \to 0} \frac{e^x - 1}{x}$

b) $\lim_{(x,y) \to (0,0)} \frac{e^{xy} - 1}{y}$

\subsection{Metodología}

Para resolver estos límites utilizaremos diferentes técnicas:

a) Para el primer límite, aplicaremos la regla de L'Hôpital, ya que tenemos una forma indeterminada $\frac{0}{0}$ cuando $x \to 0$.

b) Para el segundo límite, primero multiplicaremos por $\frac{x}{x}$ para hacer la sustitución $u = xy$, lo que nos permitirá reducir el problema a un límite de una variable y aplicar L'Hôpital.

\subsection{Resultados}
\setcounter{equation}{0}

\textbf{Inciso a}

Evaluamos el límite:
\begin{align}
\lim_{x \to 0} \frac{e^x - 1}{x}
\end{align}

Al sustituir directamente $x = 0$, obtenemos:
\begin{align}
\frac{e^0 - 1}{0} = \frac{1 - 1}{0} = \frac{0}{0}
\end{align}

Como tenemos una forma indeterminada $\frac{0}{0}$, aplicamos la regla de L'Hôpital:
\begin{align}
\lim_{x \to 0} \frac{e^x - 1}{x} &= \lim_{x \to 0} \frac{\frac{d}{dx}(e^x - 1)}{\frac{d}{dx}(x)} \\
&= \lim_{x \to 0} \frac{e^x}{1} \\
&= \lim_{x \to 0} e^x \\
&= e^0 \\
&= 1
\end{align}

\textbf{Inciso b}

Evaluamos el límite:
\begin{align}
\lim_{(x,y) \to (0,0)} \frac{e^{xy} - 1}{y}
\end{align}

Multiplicamos por $\frac{x}{x}$ para facilitar la sustitución:
\begin{align}
\lim_{(x,y) \to (0,0)} \frac{e^{xy} - 1}{y} &= \lim_{(x,y) \to (0,0)} \frac{e^{xy} - 1}{y} \cdot \frac{x}{x} \\
&= \lim_{(x,y) \to (0,0)} \frac{x(e^{xy} - 1)}{xy}
\end{align}

Hacemos la sustitución $u = xy$. Cuando $(x,y) \to (0,0)$, tenemos $u \to 0$:
\begin{align}
\lim_{(x,y) \to (0,0)} \frac{x(e^{xy} - 1)}{xy} &= \lim_{u \to 0} \frac{e^u - 1}{u} \cdot \lim_{(x,y) \to (0,0)} x
\end{align}

Del inciso a) sabemos que $\lim_{u \to 0} \frac{e^u - 1}{u} = 1$, y el $\lim_{(x,y) \to (0,0)} x = 0$.

Por lo tanto:
\begin{align}
\lim_{(x,y) \to (0,0)} \frac{e^{xy} - 1}{y} = 1 \cdot 0 = 0
\end{align}

\subsection{Discusión}
Los resultados obtenidos muestran la aplicación de diferentes técnicas para evaluar límites. En el primer caso, la regla de L'Hôpital resolvió directamente la forma indeterminada. En el segundo caso, la manipulación algebraica mediante multiplicación por $\frac{x}{x}$ y la sustitución $u = xy$ permitió reducir el problema multivariable a uno de una variable, facilitando la aplicación de resultados conocidos.

\subsection{Conclusión}
Se calcularon exitosamente ambos límites: $\lim_{x \to 0} \frac{e^x - 1}{x} = 1$ y $\lim_{(x,y) \to (0,0)} \frac{e^{xy} - 1}{y} = 0$. Las técnicas empleadas demuestran la versatilidad de métodos como la regla de L'Hôpital y la manipulación algebraica para resolver diferentes tipos de límites.

% ========================================
% SECCIÓN 7
% ========================================
\section{Problema 7}

\subsection{Enunciado}
Calcular $\nabla f(1, 1, 1)$ y $\nabla^2 f(1, 1, 1)$ donde $f(x, y, z) = (x + z)e^{x-y}$.

\subsection{Metodología}

Para resolver este problema, necesitamos:

1. Calcular el gradiente $\nabla f = \left(\frac{\partial f}{\partial x}, \frac{\partial f}{\partial y}, \frac{\partial f}{\partial z}\right)$
2. Calcular la matriz Hessiana $\nabla^2 f$ que contiene todas las derivadas parciales de segundo orden
3. Evaluar ambos en el punto $(1, 1, 1)$

Para calcular las derivadas parciales, usaremos la regla del producto y la regla de la cadena.

\subsection{Resultados}
\setcounter{equation}{0}

Primero calculamos las derivadas parciales de primer orden.

Para $\frac{\partial f}{\partial x}$, usando la regla del producto:
\begin{align}
\frac{\partial f}{\partial x} &= \frac{\partial}{\partial x}[(x + z)e^{x-y}] \\
&= \frac{\partial}{\partial x}(x + z) \cdot e^{x-y} + (x + z) \cdot \frac{\partial}{\partial x}(e^{x-y}) \\
&= 1 \cdot e^{x-y} + (x + z) \cdot e^{x-y} \cdot 1 \\
&= e^{x-y} + (x + z)e^{x-y} \\
&= e^{x-y}(1 + x + z)
\end{align}

Para $\frac{\partial f}{\partial y}$:
\begin{align}
\frac{\partial f}{\partial y} &= \frac{\partial}{\partial y}[(x + z)e^{x-y}] \\
&= (x + z) \cdot \frac{\partial}{\partial y}(e^{x-y}) \\
&= (x + z) \cdot e^{x-y} \cdot (-1) \\
&= -(x + z)e^{x-y}
\end{align}

Para $\frac{\partial f}{\partial z}$:
\begin{align}
\frac{\partial f}{\partial z} &= \frac{\partial}{\partial z}[(x + z)e^{x-y}] \\
&= \frac{\partial}{\partial z}(x + z) \cdot e^{x-y} \\
&= 1 \cdot e^{x-y} \\
&= e^{x-y}
\end{align}

Por lo tanto, el gradiente es:
\begin{align}
\nabla f(x,y,z) = \left(e^{x-y}(1 + x + z), -(x + z)e^{x-y}, e^{x-y}\right)
\end{align}

Evaluando en $(1, 1, 1)$:
\begin{align}
\nabla f(1,1,1) &= \left(e^{1-1}(1 + 1 + 1), -(1 + 1)e^{1-1}, e^{1-1}\right) \\
&= \left(e^0 \cdot 3, -2 \cdot e^0, e^0\right) \\
&= (3, -2, 1)
\end{align}

Ahora calculamos las derivadas parciales de segundo orden para formar la matriz Hessiana.

$\frac{\partial^2 f}{\partial x^2}$:
\begin{align}
\frac{\partial^2 f}{\partial x^2} &= \frac{\partial}{\partial x}[e^{x-y}(1 + x + z)] \\
&= \frac{\partial}{\partial x}(e^{x-y}) \cdot (1 + x + z) + e^{x-y} \cdot \frac{\partial}{\partial x}(1 + x + z) \\
&= e^{x-y} \cdot (1 + x + z) + e^{x-y} \cdot 1 \\
&= e^{x-y}(1 + x + z + 1) \\
&= e^{x-y}(2 + x + z)
\end{align}

$\frac{\partial^2 f}{\partial y^2}$:
\begin{align}
\frac{\partial^2 f}{\partial y^2} &= \frac{\partial}{\partial y}[-(x + z)e^{x-y}] \\
&= -(x + z) \cdot \frac{\partial}{\partial y}(e^{x-y}) \\
&= -(x + z) \cdot e^{x-y} \cdot (-1) \\
&= (x + z)e^{x-y}
\end{align}

$\frac{\partial^2 f}{\partial z^2}$:
\begin{align}
\frac{\partial^2 f}{\partial z^2} &= \frac{\partial}{\partial z}[e^{x-y}] \\
&= 0
\end{align}

$\frac{\partial^2 f}{\partial x \partial y}$:
\begin{align}
\frac{\partial^2 f}{\partial x \partial y} &= \frac{\partial}{\partial y}[e^{x-y}(1 + x + z)] \\
&= (1 + x + z) \cdot \frac{\partial}{\partial y}(e^{x-y}) \\
&= (1 + x + z) \cdot e^{x-y} \cdot (-1) \\
&= -(1 + x + z)e^{x-y}
\end{align}

$\frac{\partial^2 f}{\partial x \partial z}$:
\begin{align}
\frac{\partial^2 f}{\partial x \partial z} &= \frac{\partial}{\partial z}[e^{x-y}(1 + x + z)] \\
&= e^{x-y} \cdot \frac{\partial}{\partial z}(1 + x + z) \\
&= e^{x-y} \cdot 1 \\
&= e^{x-y}
\end{align}

$\frac{\partial^2 f}{\partial y \partial z}$:
\begin{align}
\frac{\partial^2 f}{\partial y \partial z} &= \frac{\partial}{\partial z}[-(x + z)e^{x-y}] \\
&= -\frac{\partial}{\partial z}(x + z) \cdot e^{x-y} \\
&= -1 \cdot e^{x-y} \\
&= -e^{x-y}
\end{align}

La matriz Hessiana es:
\begin{align}
\nabla^2 f(x,y,z) = \begin{pmatrix}
e^{x-y}(2 + x + z) & -(1 + x + z)e^{x-y} & e^{x-y} \\
-(1 + x + z)e^{x-y} & (x + z)e^{x-y} & -e^{x-y} \\
e^{x-y} & -e^{x-y} & 0
\end{pmatrix}
\end{align}

Evaluando en $(1, 1, 1)$:
\begin{align}
\nabla^2 f(1,1,1) &= \begin{pmatrix}
e^0(2 + 1 + 1) & -(1 + 1 + 1)e^0 & e^0 \\
-(1 + 1 + 1)e^0 & (1 + 1)e^0 & -e^0 \\
e^0 & -e^0 & 0
\end{pmatrix} \\
&= \begin{pmatrix}
4 & -3 & 1 \\
-3 & 2 & -1 \\
1 & -1 & 0
\end{pmatrix}
\end{align}

\subsection{Discusión}
El cálculo del gradiente y la matriz Hessiana requirió la aplicación sistemática de la regla del producto y la regla de la cadena para las derivadas parciales. Los resultados obtenidos son fundamentales para análisis de optimización, ya que el gradiente indica la dirección de máximo crecimiento de la función, mientras que la matriz Hessiana proporciona información sobre la curvatura local de la función en el punto evaluado.

\subsection{Conclusión}
Se calcularon exitosamente el gradiente $\nabla f(1,1,1) = (3, -2, 1)$ y la matriz Hessiana de la función $f(x, y, z) = (x + z)e^{x-y}$ en el punto $(1, 1, 1)$. Estos resultados proporcionan información completa sobre el comportamiento local de la función en dicho punto.

% ========================================
% SECCIÓN 8
% ========================================
\section{Problema 8}

\subsection{Enunciado}
Dada una función $f : \mathbb{R}^n \to \mathbb{R}$ y un punto inicial $\mathbf{x}_0 \in \mathbb{R}^n$, el método de Newton se define como sigue:
$$\mathbf{x}_{k+1} = \mathbf{x}_k - [\nabla^2 f(\mathbf{x}_k)]^{-1} \nabla f(\mathbf{x}_k), \quad k = 0, 1, 2, \ldots, K,$$

donde $\nabla f(\mathbf{x}_k)$ y $\nabla^2 f(\mathbf{x}_k)$ son respectivamente el gradiente y la matriz Hessiana de $f$ en $\mathbf{x}_k$.

Escribir un script en Python que, dado un número natural $K$, calcule el valor de $\mathbf{x}_k$ para la función $f(\mathbf{x}) = x_1^4 + 3x_2^2$. Considerar $\mathbf{x}_0 = [1, 1]$. Hacer un gráfico con los resultados.

\subsection{Solución}
La implementación y resultados de este problema se encuentran en el archivo Jupyter notebook: t01\_alejandro\_zarate\_macias.ipynb

% ========================================
% SECCIÓN 9
% ========================================
\section{Problema 9}

\subsection{Enunciado}
Sean $\mathbf{x}$ y $\mathbf{1} := (1, 1, \ldots, 1)$ ambos en $\mathbb{R}^n$. Considerar la función de Rosenbrock:
$$f(\mathbf{x}) = \sum_{i=1}^{n-1} \left[100(x_{i+1} - x_i^2)^2 + (x_i - 1)^2\right]$$

Calcular $\nabla f(\mathbf{1})$.

\subsection{Metodología}

Para calcular el gradiente de la función de Rosenbrock, necesitamos encontrar todas las derivadas parciales $\frac{\partial f}{\partial x_i}$ para $i = 1, 2, \ldots, n$.

Para encontrar las derivadas parciales, debemos analizar en qué términos de la suma aparece cada variable $x_i$. Esto nos lleva a considerar tres casos distintos según la posición de la variable en el vector:

\begin{enumerate}
    \item ($i = 1$) En este elemento es donde podemos encontrar unicamente a $x_1$.
    \item ($1 < i < n$) Aquí podemos encontrar a todos los intermedios, donde $x_i$ se puede encontrar tanto como en su respectiva evaluación, como en el paso previo.
    \item ($i=n$) pese a que la sumatoria va hasta $n-1$, es justamente en $i=n-1$ donde encontramos el único termino que incluye a $x_n$
\end{enumerate}

Los pasos a seguir son:
\begin{enumerate}
    \item[-] Identificar en qué términos aparece cada variable $x_i$ según los casos anteriores
    \item[-] Calcular la derivada parcial de cada término con respecto a $x_i$ usando la regla de la cadena
    \item[-] Evaluar el gradiente en el punto $(1, 1, \ldots, 1)$
\end{enumerate}

\subsection{Resultados}
\setcounter{equation}{0}

La función de Rosenbrock es:
\begin{align}
f(\mathbf{x}) = \sum_{i=1}^{n-1} \left[100(x_{i+1} - x_i^2)^2 + (x_i - 1)^2\right]
\end{align}

Expandiendo la suma:
\begin{align}
f(\mathbf{x}) &= 100(x_2 - x_1^2)^2 + (x_1 - 1)^2 \\
&\quad + 100(x_3 - x_2^2)^2 + (x_2 - 1)^2 \\
&\quad + 100(x_4 - x_3^2)^2 + (x_3 - 1)^2 \\
&\quad + \ldots \\
&\quad + 100(x_n - x_{n-1}^2)^2 + (x_{n-1} - 1)^2
\end{align}

Caso 1: $i = 1$

Para calcular la derivada parcial, aplicamos la regla de la cadena a cada parte:

\begin{align}
\frac{\partial f}{\partial x_1} &= \frac{\partial}{\partial x_1}\left[100(x_2 - x_1^2)^2\right] + \frac{\partial}{\partial x_1}\left[(x_1 - 1)^2\right]
\end{align}

Para el primer término, usando la regla de la cadena:
\begin{align}
\frac{\partial}{\partial x_1}\left[100(x_2 - x_1^2)^2\right] &= 100 \cdot 2(x_2 - x_1^2) \cdot \frac{\partial}{\partial x_1}(x_2 - x_1^2) \\
&= 100 \cdot 2(x_2 - x_1^2) \cdot (0 - 2x_1) \\
&= 200(x_2 - x_1^2) \cdot (-2x_1) \\
&= -400x_1(x_2 - x_1^2)
\end{align}

Para el segundo término:
\begin{align}
\frac{\partial}{\partial x_1}\left[(x_1 - 1)^2\right] &= 2(x_1 - 1) \cdot \frac{\partial}{\partial x_1}(x_1 - 1) \\
&= 2(x_1 - 1) \cdot 1 \\
&= 2(x_1 - 1)
\end{align}

Por lo tanto:
\begin{align}
\frac{\partial f}{\partial x_1} &= -400x_1(x_2 - x_1^2) + 2(x_1 - 1)
\end{align}

Evaluando en $(1, 1, \ldots, 1)$:
\begin{align}
\frac{\partial f}{\partial x_1}\bigg|_{\mathbf{1}} &= -400(1)(1 - 1^2) + 2(1 - 1) \\
&= -400(1)(1 - 1) + 2(0) \\
&= 0
\end{align}

Caso 2: $1 < i < n$

Este es el caso más complejo debido a que cada variable intermedia $x_i$ aparece en dos términos consecutivos de la suma. Específicamente, $x_i$ aparece en:

\begin{itemize}
    \item Término $(i-1)$: $100(x_i - x_{i-1}^2)^2 + (x_{i-1} - 1)^2$
    \item Término $i$: $100(x_{i+1} - x_i^2)^2 + (x_i - 1)^2$
\end{itemize}

Del término $(i-1)$ solo el primer componente contiene $x_i$:
\begin{align}
\frac{\partial}{\partial x_i}\left[100(x_i - x_{i-1}^2)^2\right] &= 100 \cdot 2(x_i - x_{i-1}^2) \cdot \frac{\partial}{\partial x_i}(x_i - x_{i-1}^2) \\
&= 100 \cdot 2(x_i - x_{i-1}^2) \cdot (1 - 0) \\
&= 200(x_i - x_{i-1}^2)
\end{align}

Del término $i$, $x_i$ aparece en ambos componentes.

Para el primer componente:

\begin{align}
\frac{\partial}{\partial x_i}\left[100(x_{i+1} - x_i^2)^2\right] &= 100 \cdot 2(x_{i+1} - x_i^2) \cdot \frac{\partial}{\partial x_i}(x_{i+1} - x_i^2) \\
&= 100 \cdot 2(x_{i+1} - x_i^2) \cdot (0 - 2x_i) \\
&= 200(x_{i+1} - x_i^2) \cdot (-2x_i) \\
&= -400x_i(x_{i+1} - x_i^2)
\end{align}

Para el segundo componente:
\begin{align}
\frac{\partial}{\partial x_i}\left[(x_i - 1)^2\right] &= 2(x_i - 1) \cdot 1 = 2(x_i - 1)
\end{align}

Sumando todas las contribuciones:
\begin{align}
\frac{\partial f}{\partial x_i} &= 200(x_i - x_{i-1}^2) + \left(-400x_i(x_{i+1} - x_i^2)\right) + 2(x_i - 1) \\
&= 200(x_i - x_{i-1}^2) - 400x_i(x_{i+1} - x_i^2) + 2(x_i - 1)
\end{align}

Evaluando en $(1, 1, \ldots, 1)$:
\begin{align}
\frac{\partial f}{\partial x_i}\bigg|_{\mathbf{1}} &= 200(1 - 1^2) - 400(1)(1 - 1^2) + 2(1 - 1) \\
&= 200(1 - 1) - 400(1)(1 - 1) + 2(0) \\
&= 0
\end{align}

Caso 3: $i = n$ 

La variable $x_n$ aparece únicamente en el último término de la suma:
$$100(x_n - x_{n-1}^2)^2 + (x_{n-1} - 1)^2$$

Observamos que $x_n$ aparece solo en el primer componente de este término. El segundo componente $(x_{n-1} - 1)^2$ no contiene $x_n$, por lo que no contribuye a la derivada.

Aplicando la regla de la cadena:
\begin{align}
\frac{\partial}{\partial x_n}\left[100(x_n - x_{n-1}^2)^2\right] &= 100 \cdot 2(x_n - x_{n-1}^2) \cdot \frac{\partial}{\partial x_n}(x_n - x_{n-1}^2) \\
&= 100 \cdot 2(x_n - x_{n-1}^2) \cdot (1 - 0) \\
&= 200(x_n - x_{n-1}^2)
\end{align}

Por lo tanto:
\begin{align}
\frac{\partial f}{\partial x_n} &= 200(x_n - x_{n-1}^2)
\end{align}

Evaluando en $\mathbf{1} = (1, 1, \ldots, 1)$:
\begin{align}
\frac{\partial f}{\partial x_n}\bigg|_{\mathbf{1}} &= 200(1 - 1^2) \\
&= 200(1 - 1) \\
&= 0
\end{align}

Resultado final:

El gradiente de la función de Rosenbrock evaluado en el $(1, 1, \ldots, 1)$ es:
\begin{align}
\nabla f(1) = (0, 0, 0, \ldots, 0)
\end{align}

\subsection{Discusión}
El resultado muestra que el punto $(1, 1, \ldots, 1)$ es un punto crítico de la función de Rosenbrock. Para este tipo de problemas se requiere de mucho análisis previo a la forma de la función, ya que al querer buscar una forma generalizada para poder derivar cada termino en función al $x_i$ en turno.

\subsection{Conclusión}
Se calculó que $\nabla f(1) = 0$, confirmando que $(1, 1, \ldots, 1)$ es un punto crítico de la función de Rosenbrock.


% ========================================
% SECCIÓN 10
% ========================================
\section{Problema 10}

\subsection{Enunciado}
Escribir un script en Python que, dado un número natural $n$ y un punto $\mathbf{x} \in \mathbb{R}^n$, aproxime $\nabla f(x)$ de la función de Rosenbrock usando diferencias finitas. Usar este programa para verificar los resultados obtenidos en el problema 9. Hacer una tabla o un gráfico mostrando el valor de la aproximación $||\nabla f(1)||$ para algunos valores decrecientes de $h$.

\subsection{Solución}
La implementación y resultados de este problema se encuentran en el archivo Jupyter notebook: t01\_alejandro\_zarate\_macias.ipynb

\end{document}