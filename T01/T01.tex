\documentclass{article}
\usepackage{graphicx}
\usepackage{amsmath}
\usepackage{amssymb}
\usepackage{amsfonts}

\title{MEMAD-T01}
\author{ALEJANDRO ZARATE MACIAS}
\date{25 de Agosto 2025}

\begin{document}

\maketitle

% ========================================
% INTRODUCCIÓN
% ========================================
\section*{Introducción}

Para esta tarea se busca la resolución de problemas relacionados con álgebra vectorial, cálculo multivariable y métodos de optimización.
La idea es utilizar lo aprendido en los videos proporcionados como material de estudio, además de otras

% ========================================
% SECCIÓN 1
% ========================================
\section{Problema 1}

\subsection{Enunciado}
Encontrar los ángulos (ambos) entre los vectores $\vec{p} = (-1, 3, 1)$ y $\vec{q} = (-2, -3, -7)$.

\subsection{Metodología}
Para la resolución de este problema, se debe utilizar la fórmula

\begin{align}
\cos(\theta) = \frac{<\vec{p},\vec{q}>}{||\vec{p}|| ||\vec{q}||}
\end{align}

donde $<\vec{p},\vec{q}>$ es el producto punto de los vectores y $||\vec{p}||$ y $||\vec{q}||$ son las normas de los vectores.
Para esto, se deben seguir los siguientes pasos:
\begin{enumerate}
    \item[-]  Calcular el producto punto $<\vec{p},\vec{q}>$.
    \item[-] Calcular las normas $||\vec{p}||$ y $||\vec{q}||$.
    \item[-] Sustituir los valores en la fórmula.
    \item[-] Despejar $\theta$.
    \item[-] Calcular los ángulos.
\end{enumerate}

\subsection{Resultados}
\setcounter{equation}{0}

Siguiendo los pasos mencionados, en la metodología, tenemos que el producto punto es:
\begin{align}
<\vec{p},\vec{q}> &= (-1)(-2) + (3)(-3) + (1)(-7) \\
&= 2 - 9 - 7 \\
&= -14
\end{align}

Ahora calculamos las normas.
Para $\vec{p}$:
\begin{align}
||\vec{p}|| &= \sqrt{(-1)^2 + 3^2 + 1^2} \\
&= \sqrt{1 + 9 + 1} \\
&= \sqrt{11}
\end{align}

Y para $\vec{q}$:
\begin{align}
||\vec{q}|| &= \sqrt{(-2)^2 + (-3)^2 + (-7)^2} \\
&= \sqrt{4 + 9 + 49} \\
&= \sqrt{62}
\end{align}

Sustituyendo los valores en la fórmula, tenemos:
\begin{align}
\cos(\theta) &= \frac{-14}{(\sqrt{11}) (\sqrt{62})} \\
&= \frac{-14}{\sqrt{682}} 
\end{align}

Despejando $\theta$, tenemos:
\begin{align}
\theta = \arccos(\frac{-14}{\sqrt{682}}) 
\end{align}

Por lo tanto:
\begin{align}
\theta_1 &= \arccos(-0.536) \\
&\approx 122.41^\circ
\end{align}

Como resultado dio mayor a $90\circ$, este es el ángulo obtuso entre los vectores.

El ángulo agudo correspondiente es:
\begin{align}
\theta_2 &= 180^\circ - 122.41^\circ \\
&= 57.59^\circ
\end{align}

\subsection{Discusión}

\subsection{Conclusión}

% ========================================
% SECCIÓN 2
% ========================================
\section{Problema 2}

\subsection{Enunciado}
Encontrar dos vectores no paralelos que sean ortogonales a $\vec{p} = (1, 1, 1)$.

\subsection{Metodología}

Para encontrar vectores ortogonales a $\vec{p} = (1, 1, 1)$, debemos encontrar vectores $\vec{v}$ tales que el producto punto $\vec{p} \cdot \vec{v} = 0$.

Si $\vec{v} = (x, y, z)$, entonces:
\begin{align}
\vec{p} \cdot \vec{v} = (1, 1, 1) \cdot (x, y, z) = x + y + z = 0
\end{align}

Para verificar que los vectores no sean paralelos, utilizaremos el concepto de proporcionalidad: dos vectores $\vec{v_1} = (a_1, b_1, c_1)$ y $\vec{v_2} = (a_2, b_2, c_2)$ son paralelos si y solo si las razones entre sus componentes correspondientes son iguales, es decir:
$$\frac{a_1}{a_2} = \frac{b_1}{b_2} = \frac{c_1}{c_2}$$

Entonces, para encontrar dos vectores no paralelos que sean ortogonales a $\vec{p}$, seguiremos estos pasos:
\begin{enumerate}
    \item[-] Elegir vectores que cumplan la condición $x + y + z = 0$.
    \item[-] Verificar que ambos sean ortogonales a $\vec{p}$.
    \item[-] Verificar que no sean paralelos usando el concepto de proporcionalidad.
\end{enumerate}

\subsection{Resultados}
\setcounter{equation}{0}

Vector 1: $\vec{v_1} = (5,-1,-4)$ \\
Verificación de ortogonalidad: \\
\begin{align}
\vec{p} \cdot \vec{v_1} &= (1, 1, 1) \cdot (5, -1, -4) \\
&= 1(5) + 1(-1) + 1(-4) \\
&= 5 - 1 - 4 \\
&= 0
\end{align}

Vector 2: $\vec{v_2} = (1, 2, -3)$ \\
Verificación de ortogonalidad: \\
\begin{align}
\vec{p} \cdot \vec{v_2} &= (1, 1, 1) \cdot (1, 2, -3) \\
&= 1(1) + 1(2) + 1(-3) \\
&= 1 + 2 - 3 \\
&= 0
\end{align}

Ahora debemos asegurarnos de que los vectores no sean paralelos, para esto usamos proporcionalidad:
\begin{align}
\frac{5}{1} &= 5 \\
\frac{-1}{2} &= -0.5 \\
\frac{-4}{-3} &\approx 1.33
\end{align}

Como $5 \neq -0.5 \neq 1.33$, los vectores no son paralelos.

\subsection{Discusión}

\subsection{Conclusión}

% ========================================
% SECCIÓN 3
% ========================================
\section{Problema 3}

\subsection{Enunciado}
Sea $n$ un número natural y $A$ como sigue:
$$A = \begin{vmatrix}
n & n+1 & n+2 \\
n+3 & n+4 & n+5 \\
n+6 & n+7 & n+8
\end{vmatrix}$$

Mostrar que $\det(A)$ permanece constante con respecto a $n$.

\subsection{Metodología}

\subsection{Resultados}
\setcounter{equation}{0}

\subsection{Discusión}

\subsection{Conclusión}

% ========================================
% SECCIÓN 4
% ========================================
\section{Problema 4}

\subsection{Enunciado}
Resolver el siguiente sistema de ecuaciones lineales usando determinantes:
\begin{align}
x + y + z &= 1 \\
2y - 2z - w &= -7 \\
x + y - z &= -3 \\
x + z + w &= 2
\end{align}

\subsection{Metodología}

\subsection{Resultados}
\setcounter{equation}{0}

\subsection{Discusión}

\subsection{Conclusión}

% ========================================
% SECCIÓN 5
% ========================================
\section{Problema 5}

\subsection{Enunciado}
Mostrar si los siguientes puntos son concíclicos o no:
$$p_1 = (-1, 6), \quad p_2 = (-1, 2), \quad p_3 = (1, 4), \quad p_4 = (0, 4 - \sqrt{3})$$

\subsection{Metodología}

\subsection{Resultados}
\setcounter{equation}{0}

\subsection{Discusión}

\subsection{Conclusión}

% ========================================
% SECCIÓN 6
% ========================================
\section{Problema 6}

\subsection{Enunciado}
Calcular los siguientes límites:

a) $\lim_{x \to 0} \frac{e^x - 1}{x}$

b) $\lim_{(x,y) \to (0,0)} \frac{e^{xy} - 1}{y}$

\subsection{Metodología}

\subsection{Resultados}
\setcounter{equation}{0}

\subsection{Discusión}

\subsection{Conclusión}

% ========================================
% SECCIÓN 7
% ========================================
\section{Problema 7}

\subsection{Enunciado}
Calcular $\nabla f(1, 1, 1)$ y $\nabla^2 f(1, 1, 1)$ donde $f(x, y, z) = (x + z)e^{x-y}$.

\subsection{Metodología}

\subsection{Resultados}
\setcounter{equation}{0}

\subsection{Discusión}

\subsection{Conclusión}

% ========================================
% SECCIÓN 9
% ========================================
\section{Problema 9}

\subsection{Enunciado}
Sean $\mathbf{x}$ y $\mathbf{1} := (1, 1, \ldots, 1)$ ambos en $\mathbb{R}^n$. Considerar la función de Rosenbrock:
$$f(\mathbf{x}) = \sum_{i=1}^{n-1} \left[100(x_{i+1} - x_i^2)^2 + (x_i - 1)^2\right]$$

Calcular $\nabla f(\mathbf{1})$.

\subsection{Metodología}

\subsection{Resultados}
\setcounter{equation}{0}

\subsection{Discusión}

\subsection{Conclusión}

\end{document}