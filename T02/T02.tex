\documentclass{article}
\usepackage{graphicx}
\usepackage{amsmath}
\usepackage{amssymb}
\usepackage{amsfonts}

\title{MEMAD-T02}
\author{ALEJANDRO ZARATE MACIAS}
\date{1 de Septiembre 2025}

\begin{document}

\maketitle

% ========================================
% INTRODUCCIÓN
% ========================================
\section*{Introducción}

Para la tarea de esta semana se busca trabajar problemas de optimización de funciones, búsqueda de puntos críticos y su clasificación como máximos o mínimos. Además de probar ciertas propiedades como lo son las matrices definidas positivas o negativas.
La idea es utilizar lo aprendido en los videos proporcionados como material de estudio, así como los libros sugeridos para el curso. Todo esto con el fin de aplicar:
\begin{itemize}
    \item Búsqueda y clasificación de puntos críticos
    \item Análisis de matrices Hessianas
    \item Matrices definidas positivas y negativas
    \item Métodos de optimización
\end{itemize}

% ========================================
% SECCIÓN 1
% ========================================
\section{Problema 1}

\subsection{Enunciado}
Para cada una de las siguientes funciones, encuentre los valores máximo y mínimo en los intervalos indicados, hallando los puntos del intervalo donde la derivada es $0$ y comparando los valores en esos puntos con los valores en los extremos.

\begin{enumerate}
  \item[(a)] $f(x) = x^{3} - x^{2} - 8x + 1$ en $[-2,\,2]$
  \item[(b)] $f(x) = \dfrac{x}{x^{2}-1}$ en $[0,\,5]$
\end{enumerate}

\subsection{Metodología}

\subsection{Resultados}
\setcounter{equation}{0}

\textbf{Inciso (a): $f(x) = x^{3} - x^{2} - 8x + 1$ en $[-2,\,2]$}

Primero calculamos la derivada:
\begin{align}
f'(x) &= 3x^{2} - 2x - 8
\end{align}

Para encontrar los puntos críticos, resolvemos $f'(x) = 0$:
\begin{align}
3x^{2} - 2x - 8 &= 0
\end{align}

Usando la fórmula cuadrática:
\begin{align}
x &= \frac{2 \pm \sqrt{4 + 96}}{6} \\
&= \frac{2 \pm \sqrt{100}}{6} \\
&= \frac{2 \pm 10}{6}
\end{align}

Por lo tanto:
\begin{align}
x_{1} &= \frac{2 + 10}{6} = \frac{12}{6} = 2 \\
x_{2} &= \frac{2 - 10}{6} = \frac{-8}{6} = -\frac{4}{3}
\end{align}

Ambos puntos críticos están dentro del intervalo $[-2, 2]$.

Ahora clasificamos los puntos críticos usando el criterio de la segunda derivada:
\begin{align}
f''(x) &= 6x - 2
\end{align}

Para $x = -\frac{4}{3}$:
\begin{align}
f''\left(-\frac{4}{3}\right) &= 6\left(-\frac{4}{3}\right) - 2 \\
&= -8 - 2 \\
&= -10 < 0
\end{align}

Como $f''(-\frac{4}{3}) < 0$, el punto $x = -\frac{4}{3}$ es un máximo local.

Para $x = 2$:
\begin{align}
f''(2) &= 6(2) - 2 \\
&= 12 - 2 \\
&= 10 > 0
\end{align}

Como $f''(2) > 0$, el punto $x = 2$ es un mínimo local.

Ahora evaluamos $f(x)$ en los puntos críticos y en los extremos:

En $x = -2$:
\begin{align}
f(-2) &= (-2)^{3} - (-2)^{2} - 8(-2) + 1 \\
&= -8 - 4 + 16 + 1 \\
&= 5
\end{align}

En $x = -\frac{4}{3}$:
\begin{align}
f\left(-\frac{4}{3}\right) &= \left(-\frac{4}{3}\right)^{3} - \left(-\frac{4}{3}\right)^{2} - 8\left(-\frac{4}{3}\right) + 1 \\
&= -\frac{64}{27} - \frac{16}{9} + \frac{32}{3} + 1 \\
&= -\frac{64}{27} - \frac{48}{27} + \frac{288}{27} + \frac{27}{27} \\
&= \frac{-64 - 48 + 288 + 27}{27} \\
&= \frac{203}{27}
\end{align}

En $x = 2$:
\begin{align}
f(2) &= 2^{3} - 2^{2} - 8(2) + 1 \\
&= 8 - 4 - 16 + 1 \\
&= -11
\end{align}


\textbf{Inciso (b): $f(x) = \dfrac{x}{x^{2}-1}$ en $[0,\,5]$}

Calculamos la derivada usando la regla del cociente:
\begin{align}
f'(x) &= \frac{(x^{2}-1)(1) - x(2x)}{(x^{2}-1)^{2}} \\
&= \frac{x^{2} - 1 - 2x^{2}}{(x^{2}-1)^{2}} \\
&= \frac{-x^{2} - 1}{(x^{2}-1)^{2}} \\
&= \frac{-(x^{2} + 1)}{(x^{2}-1)^{2}}
\end{align}

Para encontrar puntos críticos, resolvemos $f'(x) = 0$:
\begin{align}
\frac{-(x^{2} + 1)}{(x^{2}-1)^{2}} &= 0
\end{align}

Esto ocurre cuando el numerador es cero:
\begin{align}
-(x^{2} + 1) &= 0 \\
x^{2} + 1 &= 0 \\
x^{2} &= -1 \\
x &= \sqrt{-1}
\end{align}

Esta ecuación no tiene soluciones reales, por lo que no hay puntos críticos en $\mathbb{R}$.

Evaluamos en los extremos del intervalo:

En $x = 0$:
\begin{align}
f(0) &= \frac{0}{0^{2} - 1} = \frac{0}{-1} = 0
\end{align}

En $x = 5$:
\begin{align}
f(5) &= \frac{5}{5^{2} - 1} = \frac{5}{25 - 1} = \frac{5}{24}
\end{align}

\subsection{Discusión}

\subsection{Conclusión}

% ========================================
% SECCIÓN 2
% ========================================
\section{Problema 2}

\subsection{Enunciado}
Si $a_{1} < a_{2} < \cdots < a_{n}$, encuentre el valor mínimo de
\[
f(x) \;=\; \sum_{i=1}^{n} (x - a_i)^2
\]

\subsection{Metodología}

\subsection{Resultados}
\setcounter{equation}{0}

\subsection{Discusión}

\subsection{Conclusión}

% ========================================
% SECCIÓN 3
% ========================================
\section{Problema 3}

\subsection{Enunciado}
¿Cuál es la relación entre los puntos críticos de $f$ y los de $f^{2}$?

\subsection{Metodología}

\subsection{Resultados}
\setcounter{equation}{0}

\subsection{Discusión}

\subsection{Conclusión}

% ========================================
% SECCIÓN 4
% ========================================
\section{Problema 4}

\subsection{Enunciado}
Entre todos los cilindros circulares rectos de volumen fijo $V$, encuentre aquel con el área de superficie más pequeña.

\subsection{Metodología}

\subsection{Resultados}
\setcounter{equation}{0}

\subsection{Discusión}

\subsection{Conclusión}

% ========================================
% SECCIÓN 5
% ========================================
\section{Problema 5}

\subsection{Enunciado}
Demuestre que la suma de un número positivo y su recíproco es al menos $2$.

\subsection{Metodología}

\subsection{Resultados}
\setcounter{equation}{0}

\subsection{Discusión}

\subsection{Conclusión}

% ========================================
% SECCIÓN 6
% ========================================
\section{Problema 6}

\subsection{Enunciado}
Para cada una de las siguientes funciones, encuentre sus puntos críticos y clasifíquelos. Luego, use Python para crear gráficos de contorno de ellas y verificar sus resultados.

\begin{enumerate}
  \item[(a)] $f(x,y) = \ln(x^{2} + y^{2} + 1)$
  \item[(b)] $f(x,y) = x^{2} + y^{2} - x - y + 1$
  \item[(c)] $f(x,y) = e^{x}\cos(y)$
  \item[(d)] $f(x,y) = (x^{2} + y - 11)^{2} + (x + y^{2} - 7)^{2}$
\end{enumerate}

\subsection{Metodología}

\subsection{Resultados}
\setcounter{equation}{0}

\subsection{Discusión}

\subsection{Conclusión}

% ========================================
% SECCIÓN 7
% ========================================
\section{Problema 7}

\subsection{Enunciado}
Demuestre que, de todos los paralelepípedos rectangulares con una superficie dada, el cubo es el que tiene el mayor volumen.

\subsection{Metodología}

\subsection{Resultados}
\setcounter{equation}{0}

\subsection{Discusión}

\subsection{Conclusión}

% ========================================
% SECCIÓN 8
% ========================================
\section{Problema 8}

\subsection{Enunciado}
Determine si las siguientes matrices son definidas positivas.
\[
\text{(a)}\quad
A=\begin{pmatrix}
16 & -8 & -4\\
-8 & 29 & 12\\
-4 & 12 & 41
\end{pmatrix}
\qquad
\text{(b)}\quad
A=\begin{pmatrix}
9 & 0 & -8\\
6 & -5 & -2\\
-9 & 3 & 3
\end{pmatrix}
\qquad
\text{(c)}\quad
A=\begin{pmatrix}
-9 & 0 & -8\\
6 & -5 & -2\\
-9 & 3 & 3
\end{pmatrix}
\]

\subsection{Metodología}

\subsection{Resultados}
\setcounter{equation}{0}

\subsection{Discusión}

\subsection{Conclusión}

% ========================================
% SECCIÓN 9
% ========================================
\section{Problema 9}

\subsection{Enunciado}
Considere las siguientes funciones:

\begin{align}
f(\mathbf{x}) &= \sum_{i=1}^{n} x_i^{2}, \label{eq:1}\\
f(x,y) &= (x^{2}+y-11)^{2} + (x+y^{2}-7)^{2}, \label{eq:2}\\
f(\mathbf{x}) &= \sum_{i=1}^{n-1} \bigl[100(x_{i+1}-x_i^{2})^{2} + (x_i-1)^{2}\bigr]. \label{eq:3}
\end{align}

\begin{enumerate}
  \item[(a)] Crea un script en Python para aproximar sus gradientes y Hessianos, evaluados en un punto particular $\mathbf{x}$, utilizando diferencias finitas.
  \item[(b)] Encuentre sus puntos mínimos por inspección.
  \item[(c)] Usa tu script de (a) para verificar numéricamente que los puntos hallados en (b) son puntos mínimos de sus funciones correspondientes. Para las funciones (1) y (3), puedes asumir $n=5$.
\end{enumerate}

\subsection{Metodología}

\subsection{Resultados}
\setcounter{equation}{0}

\subsection{Discusión}

\subsection{Conclusión}

% ========================================
% SECCIÓN 10
% ========================================
\section{Problema 10}

\subsection{Enunciado}
Considere el método de Newton para la minimización de funciones dado por la fórmula iterativa:
\begin{equation*}
    \mathbf{x}_{k+1} \;=\; \mathbf{x}_{k} \;-\; \bigl[\nabla^{2} f(\mathbf{x}_{k})\bigr]^{-1}\,\nabla f(\mathbf{x}_{k}), k=0,1,2,\dots,K.
\end{equation*}

Codifique el método de Newton en un script de Python. Luego, úselo para optimizar las funciones del problema 9 utilizando puntos de partida suficientemente cercanos (propóngalos y proponga también un valor para $K$). Para las funciones (1) y (3), puede asumir $n = 5$. Para cada función, grafica cómo disminuye su valor a medida que el algoritmo itera (el eje horizontal corresponde a $k$ y el eje vertical a $f(x_k))$. Finalmente, imprima en su script la aproximación $x_K$ obtenida.

\subsection{Metodología}

\subsection{Resultados}
\setcounter{equation}{0}

\subsection{Discusión}

\subsection{Conclusión}

\end{document}
