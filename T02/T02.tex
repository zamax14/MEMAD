\documentclass{article}
\usepackage{graphicx}
\usepackage{amsmath}
\usepackage{amssymb}
\usepackage{amsfonts}

\title{MEMAD-T02}
\author{ALEJANDRO ZARATE MACIAS}
\date{1 de Septiembre 2025}

\begin{document}

\maketitle

% ========================================
% INTRODUCCIÓN
% ========================================
\section*{Introducción}

Para la tarea de esta semana se busca trabajar problemas de optimización de funciones, búsqueda de puntos críticos y su clasificación como máximos o mínimos. Además de probar ciertas propiedades como lo son las matrices definidas positivas o negativas.
La idea es utilizar lo aprendido en los videos proporcionados como material de estudio, así como los libros sugeridos para el curso. Todo esto con el fin de aplicar:
\begin{itemize}
    \item Búsqueda y clasificación de puntos críticos
    \item Análisis de matrices Hessianas
    \item Matrices definidas positivas y negativas
    \item Métodos de optimización
\end{itemize}

% ========================================
% SECCIÓN 1
% ========================================
\section{Problema 1}

\subsection{Enunciado}
Para cada una de las siguientes funciones, encuentre los valores máximo y mínimo en los intervalos indicados, hallando los puntos del intervalo donde la derivada es $0$ y comparando los valores en esos puntos con los valores en los extremos.

\begin{enumerate}
  \item[(a)] $f(x) = x^{3} - x^{2} - 8x + 1$ en $[-2,\,2]$
  \item[(b)] $f(x) = \dfrac{x}{x^{2}-1}$ en $[0,\,5]$
\end{enumerate}

\subsection{Metodología}

\subsection{Resultados}
\setcounter{equation}{0}

\textbf{Inciso (a): $f(x) = x^{3} - x^{2} - 8x + 1$ en $[-2,\,2]$}

Primero calculamos la derivada:
\begin{align}
f'(x) &= 3x^{2} - 2x - 8
\end{align}

Para encontrar los puntos críticos, resolvemos $f'(x) = 0$:
\begin{align}
3x^{2} - 2x - 8 &= 0
\end{align}

Usando la fórmula cuadrática:
\begin{align}
x &= \frac{2 \pm \sqrt{4 + 96}}{6} \\
&= \frac{2 \pm \sqrt{100}}{6} \\
&= \frac{2 \pm 10}{6}
\end{align}

Por lo tanto, tenemos que $x_1$ es:
\begin{align}
x_{1} &= \frac{2 + 10}{6} \\
&= \frac{12}{6} \\
&= 2 \\
\end{align}

Y por otra parte, $x_2$ es:

\begin{align}
    x_{2} &= \frac{2 - 10}{6} \\
    &= \frac{-8}{6} \\
    &= -\frac{4}{3}
\end{align}

Ambos puntos críticos están dentro del intervalo $[-2, 2]$.

Ahora clasificamos los puntos críticos usando el criterio de la segunda derivada:
\begin{align}
f''(x) &= 6x - 2
\end{align}

Para $x_1 = -\frac{4}{3}$:
\begin{align}
f''\left(-\frac{4}{3}\right) &= 6\left(-\frac{4}{3}\right) - 2 \\
&= -8 - 2 \\
&= -10 \\
&10 < 0
\end{align}

Como $f''(-\frac{4}{3}) < 0$, el punto $x = -\frac{4}{3}$ es un máximo local.

Para $x_2 = 2$:
\begin{align}
f''(2) &= 6(2) - 2 \\
&= 12 - 2 \\
&= 10 \\
&10> 0
\end{align}

Como $f''(2) > 0$, el punto $x = 2$ es un mínimo local.

Ahora evaluamos $f(x)$ en los puntos críticos y en los extremos:

En $x = -2$:
\begin{align}
f(-2) &= (-2)^{3} - (-2)^{2} - 8(-2) + 1 \\
&= -8 - 4 + 16 + 1 \\
&= 5
\end{align}

En $x = -\frac{4}{3}$:
\begin{align}
f\left(-\frac{4}{3}\right) &= \left(-\frac{4}{3}\right)^{3} - \left(-\frac{4}{3}\right)^{2} - 8\left(-\frac{4}{3}\right) + 1 \\
&= -\frac{64}{27} - \frac{16}{9} + \frac{32}{3} + 1 \\
&= -\frac{64}{27} - \frac{48}{27} + \frac{288}{27} + \frac{27}{27} \\
&= \frac{-64 - 48 + 288 + 27}{27} \\
&= \frac{203}{27}
\end{align}

En $x = 2$:
\begin{align}
f(2) &= 2^{3} - 2^{2} - 8(2) + 1 \\
&= 8 - 4 - 16 + 1 \\
&= -11
\end{align}


\textbf{Inciso (b): $f(x) = \dfrac{x}{x^{2}-1}$ en $[0,\,5]$}

Calculamos la derivada usando la regla del cociente:
\begin{align}
f'(x) &= \frac{(x^{2}-1)(1) - x(2x)}{(x^{2}-1)^{2}} \\
&= \frac{x^{2} - 1 - 2x^{2}}{(x^{2}-1)^{2}} \\
&= \frac{-x^{2} - 1}{(x^{2}-1)^{2}} \\
&= \frac{-(x^{2} + 1)}{(x^{2}-1)^{2}}
\end{align}

Para encontrar puntos críticos, resolvemos $f'(x) = 0$:
\begin{align}
\frac{-(x^{2} + 1)}{(x^{2}-1)^{2}} &= 0
\end{align}

Esto ocurre cuando el numerador es cero:
\begin{align}
-(x^{2} + 1) &= 0 \\
x^{2} + 1 &= 0 \\
x^{2} &= -1 \\
x &= \sqrt{-1}
\end{align}

Esta ecuación no tiene soluciones reales, por lo que no hay puntos críticos en $\mathbb{R}$.

Evaluamos en los extremos del intervalo:

En $x = 0$:
\begin{align}
f(0) &= \frac{0}{0^{2} - 1} = \frac{0}{-1} = 0
\end{align}

En $x = 5$:
\begin{align}
f(5) &= \frac{5}{5^{2} - 1} = \frac{5}{25 - 1} = \frac{5}{24}
\end{align}

\subsection{Discusión}


\subsection{Conclusión}


% ========================================
% SECCIÓN 2
% ========================================
\section{Problema 2}

\subsection{Enunciado}
Si $a_{1} < a_{2} < \cdots < a_{n}$, encuentre el valor mínimo de
\[
f(x) \;=\; \sum_{i=1}^{n} (x - a_i)^2
\]

\subsection{Metodología}

\subsection{Resultados}
\setcounter{equation}{0}

Para encontrar el valor mínimo de $f(x) = \sum_{i=1}^{n} (x - a_i)^2$, calculamos la derivada:

\begin{align}
    f'(x) &= 2\sum_{i=1}^{n} (x - a_i) \\
    &= 2\sum_{i=1}^{n} x - 2\sum_{i=1}^{n} a_i \\
    &= 2nx - 2\sum_{i=1}^{n} a_i
\end{align}

Para encontrar el punto crítico, igualamos $f'(x) = 0$:
\begin{align}
    2nx - 2\sum_{i=1}^{n} a_i &= 0 \\
    2nx &= 2\sum_{i=1}^{n} a_i \\
    x &= \frac{1}{n}\sum_{i=1}^{n} a_i 
\end{align}

Analizando esto, $x$ es la media de todos los valores de $a$

\begin{align}
    x = \bar{a}
\end{align}

La segunda derivada es:
\begin{align}
f''(x) = 2n > 0
\end{align}

Como $f''(x) > 0$, el punto $x = \bar{a}$ es un mínimo.

El valor mínimo de la función es:
\begin{align}
f(\bar{a}) = \sum_{i=1}^{n} (\bar{a} - a_i)^2
\end{align}

\subsection{Discusión}

\subsection{Conclusión}

% ========================================
% SECCIÓN 3
% ========================================
\section{Problema 3}

\subsection{Enunciado}
¿Cuál es la relación entre los puntos críticos de $f$ y los de $f^{2}$?

\subsection{Metodología}

\subsection{Resultados}
\setcounter{equation}{0}
Sea $g(x) = f(x)^2$. Los puntos críticos de $f$ son aquellos donde $f'(x) = 0$.

Para encontrar los puntos críticos de $g(x) = f(x)^2$, calculamos su derivada:
\begin{align}
g'(x) = 2f(x)f'(x)
\end{align}

Los puntos críticos de $g(x)$ ocurren cuando $g'(x) = 0$:
\begin{align}
g'(x) = 2f(x)f'(x) = 0
\end{align}

Esto sucede cuando:
\begin{align}
f(x) = 0 \quad \text{ó} \quad f'(x) = 0
\end{align}

Por lo tanto, los puntos críticos de $f^2$ incluyen:
\begin{itemize}
    \item Todos los puntos críticos de $f$ (donde $f'(x) = 0$)
    \item También los puntos donde $f(x) = 0$ (los ceros de $f$)
\end{itemize}

\subsection{Discusión}

\subsection{Conclusión}

% ========================================
% SECCIÓN 4
% ========================================
\section{Problema 4}

\subsection{Enunciado}
Entre todos los cilindros circulares rectos de volumen fijo $V$, encuentre aquel con el área de superficie más pequeña.

\subsection{Metodología}

\subsection{Resultados}
\setcounter{equation}{0}

Las fórmulas para un cilindro circular recto son:
\begin{align}
V &= \pi r^2 h \\
A &= 2\pi r^2 + 2\pi rh
\end{align}

De la restricción de volumen fijo $V = \pi r^2 h$, despejamos $h$:
\begin{align}
h = \frac{V}{\pi r^2}
\end{align}

Reemplazamos $h$ en la función del área:
\begin{align}
A &= 2\pi r^2 + 2\pi r \cdot \frac{V}{\pi r^2} \\
&= 2\pi r^2 + \frac{2V}{r}
\end{align}

Definimos $f(r) = A = 2\pi r^2 + \frac{2V}{r}$ como nuestra función a minimizar.

Calculamos la primera derivada:
\begin{align}
f'(r) &= 4\pi r - \frac{2V}{r^2}
\end{align}

Igualamos a cero para encontrar el punto crítico:
\begin{align}
4\pi r - \frac{2V}{r^2} &= 0 \\
4\pi r &= \frac{2V}{r^2} \\
4\pi r^3 &= 2V \\
r^3 &= \frac{V}{2\pi} \\
r &= \sqrt[3]{\frac{V}{2\pi}}
\end{align}

Calculamos la segunda derivada:
\begin{align}
f''(r) = 4\pi + \frac{4V}{r^3}
\end{align}

Evaluamos en el punto crítico:
\begin{align}
f''\left(\sqrt[3]{\frac{V}{2\pi}}\right) &= 4\pi + \frac{4V}{\left(\sqrt[3]{\frac{V}{2\pi}}\right)^3} \\
&= 4\pi + \frac{4V}{\frac{V}{2\pi}} \\
&= 4\pi + 8\pi \\
&= 12\pi > 0
\end{align}

Como $f''(r) > 0$, el punto crítico es un mínimo.

Ahora encontramos $h$ sustituyendo el valor de $r$:
\begin{align}
h &= \frac{V}{\pi r^2} \\
&= \frac{V}{\pi \left(\sqrt[3]{\frac{V}{2\pi}}\right)^2} \\
&= \frac{V}{\pi \cdot \left(\frac{V}{2\pi}\right)^{2/3}} \\
&= \frac{V^{1/3}}{\pi^{1/3}} \cdot \frac{(2\pi)^{2/3}}{V^{2/3}} \\
&= 2^{2/3} \cdot \pi^{-1/3} \cdot V^{1/3} \\
&= 2 \cdot \sqrt[3]{\frac{V}{2\pi}} \\
&= 2r
\end{align}

Por lo tanto, $h = 2r$.

Verificamos sustituyendo estos valores en la función original del área:
\begin{align}
A &= 2\pi r^2 + 2\pi rh \\
&= 2\pi r^2 + 2\pi r(2r) \\
&= 2\pi r^2 + 4\pi r^2 \\
&= 6\pi r^2
\end{align}

Sustituyendo $r = \sqrt[3]{\frac{V}{2\pi}}$:
\begin{align}
A &= 6\pi \left(\sqrt[3]{\frac{V}{2\pi}}\right)^2 \\
\end{align}

Esta es el área mínima de superficie para un cilindro de volumen fijo $V$.

\subsection{Discusión}

\subsection{Conclusión}

% ========================================
% SECCIÓN 5
% ========================================
\section{Problema 5}

\subsection{Enunciado}
Demuestre que la suma de un número positivo y su recíproco es al menos $2$.

\subsection{Metodología}

\subsection{Resultados}
\setcounter{equation}{0}

Sea $a > 0$ un número real positivo. Queremos demostrar que:
\begin{align}
a + \frac{1}{a} \geq 2
\end{align}

Partimos de la desigualdad que queremos probar:
\begin{align}
a + \frac{1}{a} \geq 2
\end{align}

Restamos $2$ de ambos lados:
\begin{align}
a + \frac{1}{a} - 2 \geq 0
\end{align}

Factorizamos la expresión del lado izquierdo:
\begin{align}
a\left(a + \frac{1}{a} - 2\right) &\geq 0 \\
a^2 - 2a + 1 &\geq 0 \\
(a - 1)^2 &\geq 0
\end{align}

El cuadrado de cualquier número real siempre es mayor o igual a cero:
\begin{align}
(a - 1)^2 \geq 0 \quad \forall a \in \mathbb{R}
\end{align}

Por lo tanto, $(a - 1)^2 \geq 0$ para todo $a \in \mathbb{R}$, y en particular para $a > 0$.


\subsection{Discusión}

\subsection{Conclusión}

% ========================================
% SECCIÓN 6
% ========================================
\section{Problema 6}

\subsection{Enunciado}
Para cada una de las siguientes funciones, encuentre sus puntos críticos y clasifíquelos. Luego, use Python para crear gráficos de contorno de ellas y verificar sus resultados.

\begin{enumerate}
  \item[(a)] $f(x,y) = \ln(x^{2} + y^{2} + 1)$
  \item[(b)] $f(x,y) = x^{2} + y^{2} - x - y + 1$
  \item[(c)] $f(x,y) = e^{x}\cos(y)$
  \item[(d)] $f(x,y) = (x^{2} + y - 11)^{2} + (x + y^{2} - 7)^{2}$
\end{enumerate}

\subsection{Metodología}

\subsection{Resultados}
\setcounter{equation}{0}

\textbf{Inciso (a): $f(x,y) = \ln(x^{2} + y^{2} + 1)$}

Calculamos las primeras derivadas parciales:
\begin{align}
f_x(x,y) &= \frac{2x}{x^2 + y^2 + 1} \\
f_y(x,y) &= \frac{2y}{x^2 + y^2 + 1}
\end{align}

Para encontrar puntos críticos, igualamos a cero:
\begin{align}
f_x(x,y) &= \frac{2x}{x^2 + y^2 + 1} = 0 \Rightarrow x = 0 \\
f_y(x,y) &= \frac{2y}{x^2 + y^2 + 1} = 0 \Rightarrow y = 0
\end{align}

El único punto crítico es $(0,0)$.

Ahora calculamos las segundas derivadas parciales para formar la matriz Hessiana. Para $f_{xx}$, derivamos $f_x$ con respecto a $x$:
\begin{align}
f_{xx}(x,y) &=  \frac{2(x^2 + y^2 + 1) - 2x(2x)}{(x^2 + y^2 + 1)^2} \\
&= \frac{2x^2 + 2y^2 + 2 - 4x^2}{(x^2 + y^2 + 1)^2} \\
&= \frac{-2x^2 + 2y^2 + 2}{(x^2 + y^2 + 1)^2} \\
&= \frac{2(-x^2 + y^2 + 1)}{(x^2 + y^2 + 1)^2}
\end{align}

Para $f_{yy}$, derivamos $f_y$ con respecto a $y$:
\begin{align}
f_{yy}(x,y) &= \frac{2(x^2 + y^2 + 1) - 2y(2y)}{(x^2 + y^2 + 1)^2} \\
&= \frac{2x^2 + 2y^2 + 2 - 4y^2}{(x^2 + y^2 + 1)^2} \\
&= \frac{2x^2 - 2y^2 + 2}{(x^2 + y^2 + 1)^2} \\
&= \frac{2(x^2 - y^2 + 1)}{(x^2 + y^2 + 1)^2}
\end{align}

Para $f_{xy}$, derivamos $f_x$ con respecto a $y$:
\begin{align}
f_{xy}(x,y) &= \frac{0 \cdot (x^2 + y^2 + 1) - 2x(2y)}{(x^2 + y^2 + 1)^2} \\
&= \frac{-4xy}{(x^2 + y^2 + 1)^2}
\end{align}

La matriz Hessiana general es:
\begin{align}
H(x,y) =  \begin{pmatrix} 
\frac{2(-x^2 + y^2 + 1)}{(x^2 + y^2 + 1)^2} & \frac{-4xy}{(x^2 + y^2 + 1)^2} \\
\frac{-4xy}{(x^2 + y^2 + 1)^2} & \frac{2(x^2 - y^2 + 1)}{(x^2 + y^2 + 1)^2}
\end{pmatrix}
\end{align}

Factorizando el factor común $\frac{2}{(x^2 + y^2 + 1)^2}$, podemos escribir:
\begin{align}
H(x,y) = \frac{2}{(x^2 + y^2 + 1)^2} \begin{pmatrix} 
1 + y^2 - x^2 & -2xy \\
-2xy & 1 + x^2 - y^2
\end{pmatrix}
\end{align}

Ahora evaluamos en el punto crítico $(0,0)$:
\begin{align}
H(0,0) &= \frac{2}{(0^2 + 0^2 + 1)^2} \begin{pmatrix} 
1 + 0^2 - 0^2 & -2(0)(0) \\
-2(0)(0) & 1 + 0^2 - 0^2
\end{pmatrix} \\
&= \frac{2}{1} \begin{pmatrix} 
1 & 0 \\
0 & 1
\end{pmatrix} \\
&= \begin{pmatrix} 
2 & 0 \\
0 & 2
\end{pmatrix}
\end{align}

Para clasificar el punto crítico, calculamos el determinante de la Hessiana:
\begin{align}
\det(H(0,0)) &= (2)(2) - (0)(0) \\
&= 4 \\
&4>0
\end{align}

Como $\det(H(0,0)) = 4 > 0$, el punto $(0,0)$ es un mínimo local.

\textbf{Inciso (b): $f(x,y) = x^{2} + y^{2} - x - y + 1$}

Calculamos las primeras derivadas parciales:
\begin{align}
f_x(x,y) &= 2x - 1 \\
f_y(x,y) &= 2y - 1
\end{align}

Para encontrar puntos críticos:
\begin{align}
f_x(x,y) = 2x - 1 &= 0 \\
x &= \frac{1}{2} \\
f_y(x,y) = 2y - 1 &= 0 \\
y &= \frac{1}{2}
\end{align}

El único punto crítico es $\left(\frac{1}{2}, \frac{1}{2}\right)$.

Las segundas derivadas parciales son:
\begin{align}
f_{xx} &= 2 \\
f_{yy} &= 2 \\
f_{xy} &= 0
\end{align}

La matriz Hessiana es:
\begin{align}
H = \begin{pmatrix} 2 & 0 \\ 0 & 2 \end{pmatrix}
\end{align}

El determinante es $\det = (2)(2) - (0)(0) = 4 > 0$, por lo que $\left(\frac{1}{2}, \frac{1}{2}\right)$ es un mínimo local.

\textbf{Inciso (c): $f(x,y) = e^{x}\cos(y)$}

Calculamos las primeras derivadas parciales:
\begin{align}
f_x(x,y) &= e^x \cos(y) \\
f_y(x,y) &= -e^x \sin(y)
\end{align}

Para encontrar puntos críticos:
\begin{align}
f_x(x,y) &= e^x \cos(y) = 0 \\
f_y(x,y) &= -e^x \sin(y) = 0
\end{align}

Como $e^x > 0$ para todo $x$ real, necesitamos:
\begin{align}
\cos(y) &= 0 \\
\sin(y) &= 0
\end{align}

No existe un valor de $y$ que satisfaga ambas condiciones simultáneamente. Por lo tanto, no hay puntos críticos.

\textbf{Inciso (d): $f(x,y) = (x^{2} + y - 11)^{2} + (x + y^{2} - 7)^{2}$ (Función de Himmelblau)}

Esta función es conocida como la función de Himmelblau. Para desarrollar las derivadas, definamos las funciones auxiliares:
\begin{align}
u(x,y) &= x^2 + y - 11 \\
v(x,y) &= x + y^2 - 7
\end{align}

Entonces $f(x,y) = u^2 + v^2$.

Calculamos las primeras derivadas parciales usando la regla de la cadena:
\begin{align}
f_x(x,y) &= 2u \cdot u_x + 2v \cdot v_x \\
&= 2(x^2 + y - 11)(2x) + 2(x + y^2 - 7)(1) \\
&= 4x(x^2 + y - 11) + 2(x + y^2 - 7)
\end{align}

\begin{align}
f_y(x,y) &= 2u \cdot u_y + 2v \cdot v_y \\
&= 2(x^2 + y - 11)(1) + 2(x + y^2 - 7)(2y) \\
&= 2(x^2 + y - 11) + 4y(x + y^2 - 7)
\end{align}

Los puntos críticos se encuentran resolviendo el sistema:
\begin{align}
f_x(x,y) &= 4x(x^2 + y - 11) + 2(x + y^2 - 7) = 0 \\
f_y(x,y) &= 2(x^2 + y - 11) + 4y(x + y^2 - 7) = 0
\end{align}

Este sistema no lineal es complejo de resolver analíticamente, pero se conocen sus cuatro soluciones:
\begin{align}
P_1 &= (3, 2) \\
P_2 &= (-2.805118, 3.131312) \\
P_3 &= (-3.779310, -3.283186) \\
P_4 &= (3.584428, -1.848126)
\end{align}

Ahora calculamos las segundas derivadas parciales. Para $f_{xx}$:
\begin{align}
f_{xx}(x,y) &= 4(x^2 + y - 11) + 4x(2x) + 2(1) \\
&= 4(x^2 + y - 11) + 8x^2 + 2 \\
&= 4x^2 + 4y - 44 + 8x^2 + 2 \\
&= 12x^2 + 4y - 42
\end{align}

Para $f_{yy}$:
\begin{align}
f_{yy}(x,y) &= 2(1) + 4(x + y^2 - 7) + 4y(2y) \\
&= 2 + 4x + 4y^2 - 28 + 8y^2 \\
&= 4x + 12y^2 - 26
\end{align}

Para $f_{xy}$:
\begin{align}
f_{xy}(x,y) &= 4x(1) + 2(2y) \\
&= 4x + 4y
\end{align}

La matriz Hessiana general es:
\begin{align}
H(x,y) = \begin{pmatrix} 
12x^2 + 4y - 42 & 4x + 4y \\
4x + 4y & 4x + 12y^2 - 26
\end{pmatrix}
\end{align}

Evaluamos la matriz Hessiana en cada punto crítico:

\textbf{Para $P_1 = (3, 2)$:}
\begin{align}
f_{xx}(3,2) &= 12(3)^2 + 4(2) - 42 \\
&= 108 + 8 - 42 = 74 \\
f_{yy}(3,2) &= 4(3) + 12(2)^2 - 26 \\
&= 12 + 48 - 26 = 34 \\
f_{xy}(3,2) &= 4(3) + 4(2) \\
&= 12 + 8 = 20
\end{align}
\begin{align}
H_1 &= \begin{pmatrix} 74 & 20 \\ 20 & 34 \end{pmatrix} \\
\det(H_1) &= (74)(34) - (20)(20) = 2516 - 400 = 2116
\end{align}
Como $\det(H_1) > 0$, el punto $(3,2)$ es un mínimo local.

\textbf{Para $P_2 = (-2.805118, 3.131312)$:}
\begin{align}
f_{xx}(-2.805118, 3.131312) &= 12(-2.805118)^2 + 4(3.131312) - 42 \\
&= 12(7.868687) + 12.525248 - 42 \\
&= 94.424244 + 12.525248 - 42 \\
&= 64.949492 \\
f_{yy}(-2.805118, 3.131312) &= 4(-2.805118) + 12(3.131312)^2 - 26 \\
&= -11.220472 + 12(9.805112) - 26 \\
&= -11.220472 + 117.661344 - 26 \\
&= 80.440872 \\
f_{xy}(-2.805118, 3.131312) &= 4(-2.805118) + 4(3.131312) \\
&= -11.220472 + 12.525248 \\
&= 1.304776
\end{align}
\begin{align}
H_2 &= \begin{pmatrix} 64.95 & 1.30 \\ 1.30 & 80.44 \end{pmatrix} \\
\det(H_2) &= (64.95)(80.44) - (1.30)(1.30) \\
&= 5225.58 - 1.69 \\
&= 5223.89
\end{align}
Como $\det(H_2) = 5223.89 > 0$, el punto $(-2.805118, 3.131312)$ es un mínimo local.

\textbf{Para $P_3 = (-3.779310, -3.283186)$:}
\begin{align}
f_{xx}(-3.779310, -3.283186) &= 12(-3.779310)^2 + 4(-3.283186) - 42 \\
&= 12(14.283191) + (-13.132744) - 42 \\
&= 171.398292 - 13.132744 - 42 \\
&= 116.265548 \\
f_{yy}(-3.779310, -3.283186) &= 4(-3.779310) + 12(-3.283186)^2 - 26 \\
&= -15.117240 + 12(10.779316) - 26 \\
&= -15.117240 + 129.351792 - 26 \\
&= 88.234552 \\
f_{xy}(-3.779310, -3.283186) &= 4(-3.779310) + 4(-3.283186) \\
&= -15.117240 - 13.132744 \\
&= -28.249984
\end{align}
\begin{align}
H_3 &= \begin{pmatrix} 116.27 & -28.25 \\ -28.25 & 88.23 \end{pmatrix} \\
\det(H_3) &= (116.27)(88.23) - (-28.25)(-28.25) \\
&= 10258.64 - 798.06 \\
&= 9460.58
\end{align}
Como $\det(H_3) = 9460.58 > 0$, el punto $(-3.779310, -3.283186)$ es un mínimo local.

\textbf{Para $P_4 = (3.584428, -1.848126)$:}
\begin{align}
f_{xx}(3.584428, -1.848126) &= 12(3.584428)^2 + 4(-1.848126) - 42 \\
&= 12(12.848125) + (-7.392504) - 42 \\
&= 154.177500 - 7.392504 - 42 \\
&= 104.784996 \\
f_{yy}(3.584428, -1.848126) &= 4(3.584428) + 12(-1.848126)^2 - 26 \\
&= 14.337712 + 12(3.414370) - 26 \\
&= 14.337712 + 40.972440 - 26 \\
&= 29.310152 \\
f_{xy}(3.584428, -1.848126) &= 4(3.584428) + 4(-1.848126) \\
&= 14.337712 - 7.392504 \\
&= 6.945208
\end{align}
\begin{align}
H_4 &= \begin{pmatrix} 104.78 & 6.95 \\ 6.95 & 29.31 \end{pmatrix} \\
\det(H_4) &= (104.78)(29.31) - (6.95)(6.95) \\
&= 3071.15 - 48.30 \\
&= 3022.85
\end{align}
Como $\det(H_4) = 3022.85 > 0$, el punto $(3.584428, -1.848126)$ es un mínimo local.

Todos los puntos críticos de la función de Himmelblau son mínimos locales con valor de función $f = 0$.

\subsection{Discusión}

\subsection{Conclusión}

% ========================================
% SECCIÓN 7
% ========================================
\section{Problema 7}

\subsection{Enunciado}
Demuestre que, de todos los paralelepípedos rectangulares con una superficie dada, el cubo es el que tiene el mayor volumen.

\subsection{Metodología}

\subsection{Resultados}
\setcounter{equation}{0}

\subsection{Discusión}

\subsection{Conclusión}

% ========================================
% SECCIÓN 8
% ========================================
\section{Problema 8}

\subsection{Enunciado}
Determine si las siguientes matrices son definidas positivas.
\[
\text{(a)}\quad
A=\begin{pmatrix}
16 & -8 & -4\\
-8 & 29 & 12\\
-4 & 12 & 41
\end{pmatrix}
\qquad
\text{(b)}\quad
A=\begin{pmatrix}
9 & 0 & -8\\
6 & -5 & -2\\
-9 & 3 & 3
\end{pmatrix}
\qquad
\text{(c)}\quad
A=\begin{pmatrix}
-9 & 0 & -8\\
6 & -5 & -2\\
-9 & 3 & 3
\end{pmatrix}
\]

\subsection{Metodología}

\subsection{Resultados}
\setcounter{equation}{0}


\textbf{Inciso (a):}
\begin{align}
    A=\begin{pmatrix} 16 & -8 & -4\\ -8 & 29 & 12\\ -4 & 12 & 41 \end{pmatrix}
\end{align}

Verificamos los menores principales:

\textbf{Primer menor principal:}
\begin{align}
M_1 = 16 > 0
\end{align}

\textbf{Segundo menor principal:}
\begin{align}
M_2 &= \det\begin{pmatrix} 16 & -8 \\ -8 & 29 \end{pmatrix} \\
&= (16)(29) - (-8)(-8) \\
&= 464 - 64 \\
&= 400 > 0
\end{align}

\textbf{Tercer menor principal (determinante completo):}
Usando la regla de Sarrus:
\begin{align}
M_3 &= \det(A) = (16)(29)(41) + (-8)(12)(-4) + (-4)(-8)(12) \\
&\quad - (-4)(29)(-4) - (12)(12)(16) - (41)(-8)(-8) \\
&= 19024 + 384 + 384 - 464 - 2304 - 2624 \\
&= 19792 - 5392 \\
&= 14400 > 0
\end{align}

Como todos los menores principales son positivos, la matriz del inciso (a) es definida positiva.

\textbf{Inciso (b):}
\begin{align}
     A=\begin{pmatrix} 9 & 0 & -8\\ 6 & -5 & -2\\ -9 & 3 & 3 \end{pmatrix}
\end{align}
Verificamos los menores principales:

\textbf{Primer menor principal:}
\begin{align}
M_1 = 9 > 0
\end{align}

\textbf{Segundo menor principal:}
\begin{align}
M_2 &= \det\begin{pmatrix} 9 & 0 \\ 6 & -5 \end{pmatrix} \\
&= (9)(-5) - (0)(6) \\
&= -45 < 0
\end{align}

Como el segundo menor principal es negativo, la matriz del inciso (b) no es definida positiva.

\textbf{Inciso (c):}
\begin{align}
     A=\begin{pmatrix} -9 & 0 & -8\\ 6 & -5 & -2\\ -9 & 3 & 3 \end{pmatrix}
\end{align}

Verificamos el primer menor principal:

\textbf{Primer menor principal:}
\begin{align}
M_1 = -9 < 0
\end{align}

Como el primer menor principal es negativo, la matriz del inciso (c) no es definida positiva.

\subsection{Discusión}

\subsection{Conclusión}

% ========================================
% SECCIÓN 9
% ========================================
\section{Problema 9}

\subsection{Enunciado}
Considere las siguientes funciones:

\begin{align}
f(\mathbf{x}) &= \sum_{i=1}^{n} x_i^{2}, \label{eq:1}\\
f(x,y) &= (x^{2}+y-11)^{2} + (x+y^{2}-7)^{2}, \label{eq:2}\\
f(\mathbf{x}) &= \sum_{i=1}^{n-1} \bigl[100(x_{i+1}-x_i^{2})^{2} + (x_i-1)^{2}\bigr]. \label{eq:3}
\end{align}

\begin{enumerate}
  \item[(a)] Crea un script en Python para aproximar sus gradientes y Hessianos, evaluados en un punto particular $\mathbf{x}$, utilizando diferencias finitas.
  \item[(b)] Encuentre sus puntos mínimos por inspección.
  \item[(c)] Usa tu script de (a) para verificar numéricamente que los puntos hallados en (b) son puntos mínimos de sus funciones correspondientes. Para las funciones (1) y (3), puedes asumir $n=5$.
\end{enumerate}

\subsection{Metodología}

\subsection{Resultados}
\setcounter{equation}{0}

\textbf{Inciso (b): Puntos mínimos por inspección}

\textbf{Función 1: $f(\mathbf{x}) = \sum_{i=1}^{n} x_i^{2}$}

Esta función es una suma de cuadrados. Cada término $x_i^2 \geq 0$, y la suma es mínima cuando cada término es mínimo.

Como $x_i^2 = 0$ únicamente cuando $x_i = 0$, el mínimo se alcanza cuando:
\begin{align}
x_i = 0 \quad \text{para todo } i = 1, 2, \ldots, n
\end{align}

El punto mínimo es:
\begin{align}
\mathbf{x}^* = (0, 0, 0, 0, 0) \quad \text{para } n = 5
\end{align}

El valor mínimo es $f(\mathbf{x}^*) = 0$.

\textbf{Función 2: $f(x,y) = (x^{2}+y-11)^{2} + (x+y^{2}-7)^{2}$}

Los cuatro puntos mínimos de esta función de Himmelblau ya fueron calculados en el Problema 6, inciso (d). En todos estos puntos el valor de la función es $f(\mathbf{x}^*) = 0$.
\begin{align}
P_1 &= (3, 2) \\
P_2 &= (-2.805118, 3.131312) \\
P_3 &= (-3.779310, -3.283186) \\
P_4 &= (3.584428, -1.848126)
\end{align}

\textbf{Función 3: $f(\mathbf{x}) = \sum_{i=1}^{n-1} \bigl[100(x_{i+1}-x_i^{2})^{2} + (x_i-1)^{2}\bigr]$}

Esta función de Rosenbrock es una suma de términos no negativos. Cada término contiene dos cuadrados:
\begin{align}
100(x_{i+1} - x_i^2)^2 &\geq 0 \\
(x_i - 1)^2 &\geq 0
\end{align}

La función es mínima cuando todos los cuadrados son cero:
\begin{align}
(x_i - 1)^2 &= 0 \quad \Rightarrow \quad x_i = 1 \\
(x_{i+1} - x_i^2)^2 &= 0 \quad \Rightarrow \quad x_{i+1} = x_i^2 = 1^2 = 1
\end{align}

Esto significa que $x_i = 1$ para todo $i$.

El punto mínimo es:
\begin{align}
\mathbf{x}^* = (1, 1, 1, 1, 1) \quad \text{para } n = 5
\end{align}

El valor mínimo es $f(\mathbf{x}^*) = 0$.

\subsection{Discusión}

\subsection{Conclusión}

% ========================================
% SECCIÓN 10
% ========================================
\section{Problema 10}

\subsection{Enunciado}
Considere el método de Newton para la minimización de funciones dado por la fórmula iterativa:
\begin{equation*}
    \mathbf{x}_{k+1} \;=\; \mathbf{x}_{k} \;-\; \bigl[\nabla^{2} f(\mathbf{x}_{k})\bigr]^{-1}\,\nabla f(\mathbf{x}_{k}), k=0,1,2,\dots,K.
\end{equation*}

Codifique el método de Newton en un script de Python. Luego, úselo para optimizar las funciones del problema 9 utilizando puntos de partida suficientemente cercanos (propóngalos y proponga también un valor para $K$). Para las funciones (1) y (3), puede asumir $n = 5$. Para cada función, grafica cómo disminuye su valor a medida que el algoritmo itera (el eje horizontal corresponde a $k$ y el eje vertical a $f(x_k))$. Finalmente, imprima en su script la aproximación $x_K$ obtenida.

\subsection{Metodología}

\subsection{Resultados}
\setcounter{equation}{0}

\subsection{Discusión}

\subsection{Conclusión}

\end{document}
