\documentclass{article}
\usepackage{graphicx}
\usepackage{amsmath}
\usepackage{amssymb}
\usepackage{amsfonts}
\usepackage{graphicx}
\usepackage{float}

\title{MEMAD-T04}
\author{ALEJANDRO ZARATE MACIAS}
\date{15 de Septiembre 2025}

\begin{document}

\maketitle

% ========================================
% INTRODUCCIÓN
% ========================================
\section*{Introducción}


% ========================================
% SECCIÓN 1
% ========================================
\section{Problema 1}

\subsection{Enunciado}

Definamos 

\begin{align*}
    s_k = x_{k+1} - x_k = \alpha_k p_k, \quad y_k = \nabla f_{k+1} - f_k.
\end{align*}

Muestre que si $\alpha_k$ y $p_k$ satisfacen las condiciones de Wolfe, entonces se cumple la siguiente desigualdad (condición de curvatura)

\begin{align}
    s_k^Ty_k > 0. 
\end{align}

\subsection{Metodología}

Para la resolución de este problema, lo que necesitamos primeramente es entender las condiciones de Wolfe y cómo es que se ven en su definición para poder encontrar una relación de estas con la condición de curvatura, o en su defecto, tratar de acomodar sus valores hasta encontrar algo que se le parezca.
Para esto primero debemos considerar lo siguiente con respecto a las condiciones de Wolfe:

\begin{itemize}
    \item El uso de constantes $0 < c_1 < c_2 < 1$
    \item Un tamaño de paso $\alpha_k > 0$
    \item Y $p_k$ como dirección de descenso es $\nabla f(x_k)^Tp_k<0$
\end{itemize}

\subsection{Resultados}
\setcounter{equation}{0}

Sabemos por la definición del problema que $s_k = \alpha_kp_k$ y que $y_k = \nabla f_{k+1} - \nabla f_k$, por lo que primero debemos modificar a $s_k^Ty_k$ para que estén en los mismos términos.

\begin{align}
    s_k^Ty_k = (\alpha_k p_k)^T(\nabla f_{k+1} - \nabla f_k)
\end{align}

A esta expresión podemos factorizar $\alpha_k$ y multiplicar los otros términos por $p_k$, y nos queda la siguiente expresión:

\begin{align}
    s_k^Ty_k = \alpha_k(\nabla f_{k+1}^Tp_k - \nabla f_k^Tp_k)
\end{align}

En las condiciones de Wolfe, la condicion de curvatura dice que:

\begin{align}
    \nabla f_{k+1}^T p_k \geq c_2 \nabla f_k^T p_k
\end{align}

Donde $0 < c_1 < c_2 < 1$. Por lo tanto, reorganizando esta desigualdad:

\begin{align}
    \nabla f_{k+1}^T p_k - \nabla f_k^T p_k &\geq c_2 \nabla f_k^T p_k - \nabla f_k^T p_k \\
    \nabla f_{k+1}^T p_k - \nabla f_k^T p_k &\geq (c_2 - 1) \nabla f_k^T p_k
\end{align}

Sustituyendo esta cota inferior en la ecuación (2):

\begin{align}
    s_k^Ty_k = \alpha_k(\nabla f_{k+1}^Tp_k - \nabla f_k^Tp_k) \geq \alpha_k(c_2 - 1) \nabla f_k^T p_k
\end{align}

Ahora analizamos los signos de cada término:

\begin{itemize}
    \item $\alpha_k > 0$ (tamaño de paso positivo)
    \item $(c_2 - 1) < 0$ (ya que $c_2 < 1$)
    \item $\nabla f_k^T p_k < 0$ (condición de dirección de descenso)
\end{itemize}

Por lo tanto:

\begin{align}
    s_k^Ty_k &\geq \alpha_k(c_2 - 1) \nabla f_k^T p_k > 0 \\
\end{align}

\subsection{Discusión}

La demostración muestra cómo las condiciones de Wolfe garantizan que la condición de curvatura $s_k^T y_k > 0$ se satisfaga. Esta propiedad es fundamental en los métodos quasi-Newton, ya que asegura que las aproximaciones de la matriz Hessiana mantengan la definición positiva, lo cual es crucial para la convergencia del algoritmo.

\subsection{Conclusión}

Hemos demostrado exitosamente que si $\alpha_k$ y $p_k$ satisfacen las condiciones de Wolfe, entonces se cumple la condición de curvatura $s_k^T y_k > 0$. Esta demostración se basa en la aplicación directa de la segunda condición de Wolfe y el análisis cuidadoso de los signos de los términos involucrados en la expresión final.

% ========================================
% SECCIÓN 2
% ========================================
\section{Problema 2}

\subsection{Enunciado}

Considere la ecuación secante
\begin{equation}\tag{2}
    B_{k+1}\, s_k = y_k,
\end{equation}
donde se asume que \(B_k > 0\) y \(B_k = B_k^{\top}\). Recuerde que si (1) se cumple, entonces (2) siempre tiene solución. De hecho, ese sistema tiene un número infinito de soluciones, ya que los grados de libertad son las entradas de \(B_k\). ¿Cuántos grados de libertad tiene (2)? ¿Cuántas condiciones impuestas (ecuaciones) tiene (2)?

\subsection{Metodología}

Para resolver este problema, analizaremos la estructura de la matriz simétrica $B_{k+1}$ de dimensión $n \times n$.

\begin{itemize}
    \item Contar las entradas independientes de una matriz simétrica
    \item Determinar el número de ecuaciones que impone $B_{k+1} s_k = y_k$
    \item Calcular la diferencia entre grados de libertad y condiciones
\end{itemize}

\subsection{Resultados}
\setcounter{equation}{0}

Para una matriz simétrica $B_{k+1}$ de $n \times n$, las entradas independientes son: 
\begin{itemize}
    \item $n$ valores de la diagonal
    \item $\frac{n(n-1)}{2}$ del triángulo superior
\end{itemize}

Por lo tanto:

\begin{align}
    \text{Grados de libertad} = n + \frac{n(n-1)}{2} = \frac{n(n+1)}{2}
\end{align}

La ecuación secante $B_{k+1} s_k = y_k$ representa $n$ ecuaciones lineales.

Considerando que $s_k$ es un vector dado y la estructura de la ecuación secante, el número de condiciones independientes efectivas es:

\begin{align}
    \text{Condiciones impuestas} = \frac{n(n-1)}{2}
\end{align}

\subsection{Discusión}

El análisis muestra que tenemos más grados de libertad que condiciones, lo que explica por qué la ecuación secante tiene infinitas soluciones. Esta propiedad es fundamental en métodos quasi-Newton para justificar la necesidad de criterios adicionales.

\subsection{Conclusión}

La ecuación secante tiene $\frac{n(n+1)}{2}$ grados de libertad y $\frac{n(n-1)}{2}$ condiciones impuestas, resultando en un sistema subdeterminado con múltiples soluciones.

% ========================================
% SECCIÓN 3
% ========================================
\section{Problema 3}

\subsection{Enunciado}

Calcular la norma de Frobenius de las siguientes matrices:

\begin{itemize}
    \item[(a)] \[A=\begin{pmatrix}
                        16 & -8 & -4\\
                        -8 & 29 & 12\\
                        -4 & 12 & 41
                    \end{pmatrix}.\]
    \item[(b)] \[A=\begin{pmatrix}
                        9 & 0 & -8\\
                        6 & -5 & -2\\
                        -9 & 3 & 3
                    \end{pmatrix}.\]
    \item[(c)] \[A=\begin{pmatrix}
                        -9 & 0 & -8\\
                        6 & -5 & -2\\
                        -9 & 3 & 3
                    \end{pmatrix}.\]
\end{itemize}

\subsection{Metodología}

Para calcular la norma de Frobenius de una matriz $A$ de dimensión $m \times n$, utilizaremos la fórmula:

\begin{align}
    ||A||_F = \sqrt{\sum_{i=1}^{m} \sum_{j=1}^{n} |a_{ij}|^2}
\end{align}
\subsection{Resultados}
\setcounter{equation}{0}

Para la matriz (a), calculamos:

\begin{align}
    ||A||_F = \sqrt{16^2 + (-8)^2 + (-4)^2 + (-8)^2 + 29^2 + 12^2 + (-4)^2 + 12^2 + 41^2}
\end{align}

\begin{align}
    ||A||_F &= \sqrt{256 + 64 + 16 + 64 + 841 + 144 + 16 + 144 + 1681} \\
    &= \sqrt{3226} \\
    &\approx 56.797
\end{align}

Para la matriz (b), calculamos:

\begin{align}
    ||A||_F = \sqrt{9^2 + 0^2 + (-8)^2 + 6^2 + (-5)^2 + (-2)^2 + (-9)^2 + 3^2 + 3^2}
\end{align}

\begin{align}
    ||A||_F &= \sqrt{81 + 0 + 64 + 36 + 25 + 4 + 81 + 9 + 9} \\
    &= \sqrt{309} \\
    &\approx 17.578
\end{align}

Para la matriz (c), solo cambia el signo del primer elemento, por lo que la norma es la misma:

\begin{align}
    ||A||_F &= \sqrt{(-9)^2 + 0^2 + (-8)^2 + 6^2 + (-5)^2 + (-2)^2 + (-9)^2 + 3^2 + 3^2} \\
    &\approx 17.578
\end{align}

\subsection{Discusión}

Los resultados muestran que las matrices (b) y (c) tienen la misma norma de Frobenius debido a que la norma no se ve afectada por el signo de los elementos, ya que todos se elevan al cuadrado. La matriz (a) tiene una norma considerablemente mayor debido a sus valores más grandes.

\subsection{Conclusión}

Las normas de Frobenius calculadas son: matriz (a) = 56.797, matriz (b) = 17.578, y matriz (c) = 17.578, confirmando que el cambio de signo en un elemento no afecta la norma de Frobenius.

% ========================================
% SECCIÓN 4
% ========================================
\section{Problema 4}

\subsection{Enunciado}

\subsection{Metodología}

\subsection{Resultados}
\setcounter{equation}{0}

\subsection{Discusión}

\subsection{Conclusión}

% ========================================
% SECCIÓN 5
% ========================================
\section{Problema 5}

\subsection{Enunciado}

\subsection{Metodología}

\subsection{Resultados}
\setcounter{equation}{0}

\subsection{Discusión}

\subsection{Conclusión}

% ========================================
% SECCIÓN 6
% ========================================
\section{Problema 6}

\subsection{Enunciado}

\subsection{Metodología}

\subsection{Resultados}
\setcounter{equation}{0}

\subsection{Discusión}

\subsection{Conclusión}

% ========================================
% SECCIÓN 7
% ========================================
\section{Problema 7}

\subsection{Enunciado}

\subsection{Metodología}

\subsection{Resultados}
\setcounter{equation}{0}

\subsection{Discusión}

\subsection{Conclusión}

% ========================================
% SECCIÓN 8
% ========================================
\section{Problema 8}

\subsection{Enunciado}

\subsection{Metodología}

\subsection{Resultados}
\setcounter{equation}{0}

\subsection{Discusión}

\subsection{Conclusión}

% ========================================
% SECCIÓN 9
% ========================================
\section{Problema 9}

\subsection{Enunciado}

\subsection{Metodología}

\subsection{Resultados}
\setcounter{equation}{0}

\subsection{Discusión}

\subsection{Conclusión}

% ========================================
% SECCIÓN 10
% ========================================
\section{Problema 10}

\subsection{Enunciado}

\subsection{Metodología}

\subsection{Resultados}
\setcounter{equation}{0}

\subsection{Discusión}

\subsection{Conclusión}

\end{document}
