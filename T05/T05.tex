\documentclass{article}
\usepackage{graphicx}
\usepackage{amsmath}
\usepackage{amssymb}
\usepackage{amsfonts}
\usepackage{graphicx}
\usepackage{float}

\title{MEMAD-T05}
\author{ALEJANDRO ZARATE MACIAS}
\date{22 de Septiembre 2025}

\begin{document}

\maketitle

% ========================================
% INTRODUCCIÓN
% ========================================
\section*{Introducción}

% ========================================
% SECCIÓN 1
% ========================================
\section{Problema 1}

\subsection{Enunciado}

Considere la siguiente función
\begin{align*}
    f(x) = 2^{cos(x^2)}, \quad x \in (-\pi,\pi).
\end{align*}
A continuación, haga lo siguiente:

\begin{itemize}
    \item Calcule 500 pares de datos de muestra $D = {x_i,f(x_i)}$ donde los $x_i$ están equiespaciados.
    \item Realice un gráfico de $D$.
    \item Considere un modelo polinomial de grado $\mathcal{N}$:
        \begin{equation*}
            h(x_i, \theta) = \hat{y_i} = \sum^{\mathcal{N}}_{\ell=0}\theta_{\ell}x^{\ell}_{i}, \quad \theta=(\theta_{0}, \theta_{1}, \theta_{2}, ...,\theta_{\mathcal{N}}) \in \mathbb{R}^{\mathcal{N}+1}.
        \end{equation*}
    \item Considere una función de pérdida y costo de MSE:
        \begin{align*}
            Loss(y_i, \hat{y_i}) = (y_i - \hat{y_i})^2, \\
            Cost(\theta;D) = \frac{1}{m}\sum^{m}_{i=1}Loss(y_i, \hat{y_i})
        \end{align*}
    \item Codifique un script en Python para resolver el problema de optimización asociado utilizando un algoritmo de búsqueda lineal determinista. Puede probar $\mathcal{N}$ = 8, por ejemplo.
    \item Haz una gráfica de la solución contra $f(x_i)$ y muestra en otra gráfica cómo el costo disminuye a medida que pasan las iteraciones.
\end{itemize}

\subsection{Metodología}

Para resolver este problema, primero se deben implementar las funciones definidas en el enunciado ($f(x)$, $h$, $loss$ y $cost$).

Una vez implementadas estas funciones, procederemos a utilizar un algoritmo de búsqueda lineal determinista. Para este caso, emplearemos el método de Gauss-Newton implementado en tarea 04, pero ya con las correcciones realizadas en el algoritmo, a diferencia de la entrega pasada (en mi caso).

El procedimiento consistirá en:
\begin{enumerate}
    \item Generar 500 puntos equiespaciados en el intervalo $(-\pi, \pi)$ utilizando numpy y la función "linespace".
    \item Calcular los valores de $f(x_i)$ para cada punto.
    \item Aplicar el algoritmo de Gauss-Newton para encontrar los parámetros óptimos $\theta$.
    \item Finalmente graficar los resultados y el comportamiento de la función de costo.
\end{enumerate}

\subsection{Resultados}
\setcounter{equation}{0}

La Figura \ref{fig:fx} muestra la función original $f(x) = 2^{\cos(x^2)}$ evaluada en los 500 puntos de muestra equiespaciados en el intervalo $(-\pi, \pi)$.

\begin{figure}[H]
    \centering
    \includegraphics[width=0.8\textwidth]{images/1_fx.png}
    \caption{Función original $f(x) = 2^{\cos(x^2)}$ con 500 puntos de muestra}
    \label{fig:fx}
\end{figure}

Para los parámetros de optimización: $\mathcal{N}=8$, tasa de aprendizaje inicial $\alpha = 0.01$, 2000 iteraciones máximas y tolerancia de $1 \times 10^{-8}$, obtuvimos que el algoritmo de Gauss-Newton alcanzó su punto óptimo en 170 iteraciones.

\begin{figure}[H]
    \centering
    \includegraphics[width=0.8\textwidth]{images/1_gauss_newton_N8.png}
    \caption{Resultados del ajuste polinomial usando Gauss-Newton con $\mathcal{N}=8$}
    \label{fig:gauss_newton_results}
\end{figure}

\subsection{Discusión}

Después de haber podido implementar de manera exitosa el método de Gauss-Newton para esta tarea, a diferencia de la implementación pasada, logramos la convergencia de la función de manera satisfactoria. 

Para el caso del polinomio de grado $\mathcal{N}=8$, aunque la función resultante no hace un "fit" exacto a la función original, se asemeja considerablemente a ella. Esto es esperado dado que estamos aproximando una función con un polinomio de grado relativamente bajo a comparacion de la original ($f(x)$).

La convergencia en 170 iteraciones indica que el algoritmo fue eficiente para este problema particular, y la tolerancia establecida fue adecuada para obtener una solución de buena calidad.

\subsection{Conclusión}

Logramos resolver exitosamente el problema de optimización utilizando el método de Gauss-Newton. La implementación mejorada del algoritmo nos permitió obtener una convergencia adecuada y un ajuste polinomial que aproxima razonablemente bien la función objetivo $f(x) = 2^{\cos(x^2)}$. Los resultados demuestran que el enfoque determinista de búsqueda lineal es efectivo para este tipo de problemas de regresión polinomial, logrando una solución satisfactoria en un número bajo de iteraciones.


% ========================================
% SECCIÓN 2
% ========================================
\section{Problema 2}

\subsection{Enunciado}

Intenta encontrar un tamaño más adecuado para el modelo paramétrico del problema anterior resolviendo de nuevo varios valores de $\mathcal{N}$. Crea gráficos que muestren tus hallazgos. A partir de aquí, puedes fijar este valor de N para los problemas restantes.

\subsection{Metodología}

Para encontrar el tamaño más adecuado del modelo paramétrico, tomaremos la implementación previa del algoritmo de Gauss-Newton y procederemos a iterar con distintos valores de $\mathcal{N}$ para determinar cuál puede ser el más óptimo para este problema.

Después de varias pruebas experimentales, se descubrió que los puntos donde la función resultante llega a adquirir diferentes formas es con los valores $\mathcal{N} = [2, 4, 8, 12, 16, 20]$. Estos valores nos permitirán observar cómo evoluciona la aproximación polinomial conforme aumenta el grado.

El procedimiento consistirá en:
\begin{enumerate}
    \item Aplicar el algoritmo de Gauss-Newton para cada valor de $\mathcal{N}$
    \item Agrupar y mostrar las gráficas del resultado de la optimización de los valores $\theta$
    \item Comparar el avance del costo para cada grado de polinomio
\end{enumerate}

\subsection{Resultados}
\setcounter{equation}{0}

Para los valores de $\mathcal{N} = [2, 4, 8, 12, 16, 20]$ obtuvimos que casi todas las optimizaciones convergen en un número similar de iteraciones, pero los resultados son considerablemente distintos entre sí.

\begin{figure}[H]
    \centering
    \includegraphics[width=0.8\textwidth]{images/2_comparasion_Ns.png}
    \caption{Comparación de ajustes polinomiales para diferentes grados $\mathcal{N}$}
    \label{fig:comparison_N}
\end{figure}

\begin{figure}[H]
    \centering
    \includegraphics[width=0.8\textwidth]{images/2_comparasion_Ns_costs.png}
    \caption{Evolución del costo para diferentes grados de polinomio}
    \label{fig:comparison_costs}
\end{figure}

Los resultados muestran que conforme aumenta el grado del polinomio, la aproximación a la función original mejora significativamente, especialmente para $\mathcal{N} \geq 12$.

\subsection{Discusión}

Pese a que $\mathcal{N} = 20$ resultó ser el valor más apegado a la función original, es considerablemente más costoso computacionalmente calcularlo. Por el contrario, $\mathcal{N} = 12$ representa un polinomio no tan grande pero con una aproximación bastante decente.

Los polinomios de grado bajo ($\mathcal{N} = 2, 4$) muestran aproximaciones muy limitadas que no capturan la complejidad de la función original. Los grados intermedios ($\mathcal{N} = 8, 12$) ofrecen un buen balance entre precisión y eficiencia computacional, mientras que los grados altos ($\mathcal{N} = 16, 20$) proporcionan la mejor aproximación a costa de mayor complejidad.

\subsection{Conclusión}

Logramos identificar que $\mathcal{N} = 12$ representa el valor óptimo para el modelo paramétrico, ofreciendo un equilibrio adecuado entre precisión de aproximación y eficiencia computacional. Pese a lo anteriormente mencionado, para los siguientes problemas se estará utilizando $\mathcal{N} = 20$ simplemente para comparar los siguientes algoritmos en un terreno más demandante e intentar exprimir sus ventajas lo más que se pueda.

% ========================================
% SECCIÓN 3
% ========================================
\section{Problema 3}

\subsection{Enunciado}

Cree un script para resolver el Problema 1 utilizando un gradiente descendente de Minibach con una tasa de aprendizaje constante $\alpha$. Genere una gráfica de la solución en función de $f(x_i)$ y muestre en otra gráfica cómo disminuye el coste a medida que transcurren las iteraciones.

\subsection{Metodología}

Para resolver este problema utilizaremos el algoritmo de descenso de gradiente estocástico con enfoque en Minibatch como alternativa al método determinista de Gauss-Newton empleado anteriormente para reducir un poco la carga computacional sobre las iteraciones.

Implementaremos las mismas funciones base ($f(x)$, $h$, $loss$ y $cost$) junto con una función de gradiente específica para SGD, ademas de una función enfocada a dividir nuestro dataset en "batches" para poder iterar sobre ellos.

El procedimiento consistirá en:
\begin{enumerate}
    \item Utilizar los mismos 500 puntos equiespaciados del Problema 1
    \item Implementar el algoritmo SGD Minibatch con tamaño de lote de 50 muestras
    \item Aplicar una tasa de aprendizaje constante $\alpha = 0.1$
    \item Usar $\mathcal{N} = 20$ (basado en los resultados del Problema 2)
    \item Ejecutar el algoritmo por 2000 épocas máximo
    \item Graficar los resultados y la evolución del costo
\end{enumerate}

\subsection{Resultados}
\setcounter{equation}{0}

Para los parámetros establecidos: $\mathcal{N} = 20$, $\alpha = 0.1$, tamaño de lote = 50, y 2000 épocas máximas, obtuvimos una convergencia satisfactoria del algoritmo SGD Minibatch.

\begin{figure}[H]
    \centering
    \includegraphics[width=0.8\textwidth]{images/3_sgd_minibatch.png}
    \caption{Resultados del ajuste polinomial usando SGD Minibatch con $\mathcal{N}=20$}
    \label{fig:sgd_results}
\end{figure}

\subsection{Discusión}

El método SGD Minibatch demostró ser efectivo para este problema de regresión polinomial. A diferencia del método determinista de Gauss-Newton, SGD no aproxima tan bien la función como en el caso de Gauss Newton, e incluso hizo uso de todas las iteraciones indicadas sin alcanzar la tolerancia indicada. Pese a esto, los resultados son bastante buenos y en velocidad sí se percibe que, pese al hacer más iteraciones, SGD es bastante más rápido que Gauss-Newton.

La tasa de aprendizaje de 0.1 resultó adecuada para lograr convergencia sin causar inestabilidad en el entrenamiento, y el tamaño de lote de 50 proporcionó un buen balance entre eficiencia computacional y estabilidad.

\subsection{Conclusión}

Logramos implementar exitosamente el algoritmo SGD Minibatch para resolver el mismo problema de aproximación polinomial. El método estocástico proporcionó resultados comparables al método determinista, demostrando la viabilidad de enfoques estocásticos para problemas de regresión.

% ========================================
% SECCIÓN 4
% ========================================
\section{Problema 4}

\subsection{Enunciado}

Repita el Problema 3 usando Adam.

\subsection{Metodología}

Para este problema implementaremos el algoritmo ADAM, el perfecto balance entre el manejo de datos de minibatch, momento y RMSprop para el ajuste de theta. Los pasos serán los siguientes:

\begin{enumerate}
    \item Utilizar los mismos 500 puntos equiespaciados de problemas anteriores
    \item Implementar ADAM con parámetros estándar: $\beta_1 = 0.9$, $\beta_2 = 0.999$
    \item Aplicar $\alpha = 0.1$, $\mathcal{N} = 20$, tamaño de lote = 50, 2000 épocas máximo
    \item Comparar resultados con SGD Minibatch
\end{enumerate}

\subsection{Resultados}
\setcounter{equation}{0}

Para los parámetros establecidos obtuvimos una convergencia notablemente superior con ADAM comparado con SGD Minibatch.

\begin{figure}[H]
    \centering
    \includegraphics[width=0.8\textwidth]{images/4_adam.png}
    \caption{Resultados del ajuste polinomial usando ADAM con $\mathcal{N}=20$}
    \label{fig:adam_results}
\end{figure}

\subsection{Discusión}

ADAM demostró ventajas claras sobre SGD Minibatch en términos de calidad del ajuste final y mejor aproximación a la función original. Sin embargo, mostró un comportamiento interesante donde la función de costo presenta picos iniciales que luego se corrigen, contrastando con SGD que mantuvo una evolución más constante sin picos pronunciados.

\subsection{Conclusión}

Logramos implementar exitosamente ADAM obteniendo resultados superiores a SGD Minibatch en estabilidad y calidad de convergencia. Los resultados confirman las ventajas de los métodos adaptativos sobre gradiente tradicional, estableciendo una base sólida para las comparaciones del siguiente problema.


% ========================================
% SECCIÓN 5
% ========================================
\section{Problema 5}

\subsection{Enunciado}

Compara las soluciones y el rendimiento de los tres algoritmos que utilizaste para resolver el problema 1. Escribe tus conclusiones y haz algunos gráficos.

\subsection{Metodología}

Para este problema reutilizaremos todas las clases ya implementadas (GaussNewton, SGD y Adam) para comparar los resultados de los tres algoritmos en una misma gráfica y evaluar sus diferencias en términos de calidad de ajuste y comportamiento de convergencia.

\subsection{Resultados}
\setcounter{equation}{0}

\begin{figure}[H]
    \centering
    \includegraphics[width=0.8\textwidth]{images/5_comparation_algorithms.png}
    \caption{Comparación de los tres algoritmos: Gauss-Newton, SGD Minibatch y ADAM}
    \label{fig:comparison_algorithms}
\end{figure}

\subsection{Discusión}

Como hemos observado desde el inicio, Gauss-Newton representa el mejor ajuste a la función original a costa de un mayor costo computacional. Por el contrario, SGD y ADAM son considerablemente más rápidos, aunque generalizan más el resultado sin parecerse completamente a la función original.

Entre los métodos, ADAM se destaca al poder disminuir más la función de costo que SGD y lograr un mejor parecido a la función original. Esto confirma las ventajas de los métodos adaptativos en términos de calidad de convergencia, manteniendo la eficiencia computacional.

\subsection{Conclusión}

La comparación confirma que existe un balance entre precisión y eficiencia computacional. Gauss-Newton ofrece la mayor precisión pero con mayor costo, mientras que los métodos estocásticos proporcionan soluciones más rápidas con calidad aceptable. ADAM destaca como el mejor entre ambos enfoques, combinando eficiencia computacional con mejor calidad de ajuste que SGD tradicional.


% ========================================
% SECCIÓN 6
% ========================================
\section{Problema 6}

\subsection{Enunciado}

Hasta ahora has resuelto un problema sin ruido, es decir,

\begin{equation*}
    f(x)=2^{cos(x^2)}+\mathcal{X}, \quad x \in (-\pi,\pi), \mathcal{X} \sim \mathcal{N}(0,\sigma^{2}),
\end{equation*}

Donde solo se ha considerado el caso con $\sigma^2=0$. Repita los ejercicios 1 a 5 para los niveles de ruido $\sigma^2 = 0.05,0.1,0.5$. No olvide anotar todas sus conclusiones y reflexiones.

\subsection{Metodología}

Para evaluar el comportamiento de los algoritmos bajo condiciones de ruido, implementamos una nueva función "f\_noise(x)" que altera los valores de $y$ en función de $x$ y el valor de $\sigma$ indicado.

Aplicamos los tres algoritmos (Gauss-Newton, SGD Minibatch y ADAM) para cada nivel de ruido propuesto: $\sigma^2 = 0.05, 0.1, 0.5$, mediante un loop que compare todos los métodos según $\sigma^2$, pero en este caso se optará por utilizar $\mathcal{N}=12$ para disminuir un poco la carga computacional de estos ciclos iterativos donde debemos evalur muchas veces los algoritmos.

\subsection{Resultados}
\setcounter{equation}{0}

\begin{figure}[H]
    \centering
    \includegraphics[width=0.8\textwidth]{images/6_sigma_05.png}
    \caption{Comparación de algoritmos con ruido $\sigma^2 = 0.05$}
    \label{fig:noise_005}
\end{figure}

\begin{figure}[H]
    \centering
    \includegraphics[width=0.8\textwidth]{images/6_sigma_1.png}
    \caption{Comparación de algoritmos con ruido $\sigma^2 = 0.1$}
    \label{fig:noise_01}
\end{figure}

\begin{figure}[H]
    \centering
    \includegraphics[width=0.8\textwidth]{images/6_sigma_5.png}
    \caption{Comparación de algoritmos con ruido $\sigma^2 = 0.5$}
    \label{fig:noise_05}
\end{figure}

\subsection{Discusión}

Los resultados muestran que conforme aumenta el nivel de ruido, todos los algoritmos enfrentan mayor dificultad para aproximar la función original. Sin embargo, mantienen patrones de comportamiento consistentes con las observaciones previas.

Gauss-Newton continúa proporcionando el mejor ajuste incluso en presencia de ruido, aunque su ventaja se reduce conforme aumenta $\sigma^2$. Los métodos estocásticos (SGD y ADAM) muestran mayor robustez al ruido, manteniendo aproximaciones razonables incluso con $\sigma^2 = 0.5$, siendo ADAM nuevamente quien más destaca.

\subsection{Conclusión}

La evaluación con ruido confirma que los algoritmos mantienen sus características distintivas bajo condiciones mas complejas. Gauss-Newton conserva su precisión superior, mientras que los métodos estocásticos ofrecen mayor robustez. El ruido afecta la calidad del ajuste en todos los casos, pero ADAM vuelve a mostrarse como el método más equilibrado para escenarios con datos ruidosos.

% ========================================
% SECCIÓN 7
% ========================================
\section{Problema 7}

\subsection{Enunciado}

Explique con sus propias palabras lo siguiente:

\begin{itemize}
    \item[(a)] Las diferencias y similitudes entre el descenso de gradiente estocástico, el descenso más pronunciado y el descenso de gradiente en minilotes.
    \item[(b)] Algunos ejemplos del uso de los tres últimos algoritmos citados (es decir, explique cómo identificar cuándo se debe utilizar un algoritmo entre otros).
    \item[(c)] Algunos ejemplos del uso de un enfoque determinista (Newton, descenso más pronunciado, BFGS, Gauss-Newton, etc.) y un enfoque estocástico (descenso de gradiente estocástico, descenso de gradiente en minilotes, Adam, etc.), es decir, explique cómo identificar cuándo se debe utilizar un enfoque particular sobre otro.

\end{itemize}

\subsection{Metodología}

\subsection{Resultados}
\setcounter{equation}{0}

\subsection{Discusión}

\subsection{Conclusión}

\end{document}
